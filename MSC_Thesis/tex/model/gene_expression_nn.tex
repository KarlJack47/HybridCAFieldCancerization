\providecommand{\main}{../..}
\documentclass[\main/thesis.tex]{subfiles}

\begin{document}

\section{Gene Expression Neural Network}

We consider $G {\in} \mathbb{N}$ genes that are biomarkers to the considered 
cancer type. The gene expression of each gene is represented by the function
$$
e_j(x, y, t) {:=} u_j(x, y, t) {-} d_j(x, y, t), j {=} 1, 2, ..., G,
$$
where $u_j(x, y, t) {\in} \mathbb{R}_+$ is the upregulation of the gene $j$ and
$d_j(x, y, t) {\in} \mathbb{R}_+$ is the downregulation of the gene $j$. The 
gene expression of each gene changes over time based upon a simple multilayer 
perceptron (MLP). The input of the MLP is the vector
$$
\vec{X}(x, y, t) {:=} 
  \{
    \{ c_i(x, y, t {-} 1) \}_{i {=} 1,...,C},
    \alpha(x, y, t {-} 1)
  \} 
{\in} \mathbb{R}^{C {+} 1},
$$
where $c_i(x, y, t)$ are the carcinogen concentrations and $\alpha(x, y, t)$ is 
the age of the cell. This choice is such that changes in gene expression is based 
upon the carcinogens in the environment of the cell and the age, are 
essentially means we are looking at the effects of the carcinogens
and replication errors as a cell ages. The output of the MLP is given by
$$
\vec{Y}(x, y, t) {:=} \{ \{ \delta_j(x, y, t) \}_{j {=} 1, 2, ..., G} \}
{\in} \mathbb{R}^{G},
$$
where $\delta(x, y, t)_j$ is the computed change in gene expression for gene 
$j$. 

$Y(x, y, t)$ is computed using some matrix multiplication, addition and 
application of a non-linear transform. The hidden layer is computed by
$$
\vec{h}(x, y, t) {:=} \gamma(W_x \vec{X}(x, y, t))
{\in} \mathbb{R}^{G},
$$
where
$$
\gamma(\xi) {:=} \frac{\xi}
                      {\sqrt{1 {+} \nu \xi^2}},
{\in} \left(
        \frac{\minus 1}
             {\sqrt{\nu}},
        \frac{1}
             {\sqrt{\nu}}
      \right) 
$$
is the non-linear transform (also known as an activation function) that is 
applied element wise to a vector and $W_x {\in} \mathbb{R}^{G {\times} C {+} 1}$ 
is a weight matrix. Note that the activation function is chosen so to 
control the amount of change that can occur for each genes' expression by 
the parameter $\nu$. After the hidden layer is computed the output is 
computed by
$$
\vec{Y}(x, y, t) {=} \left| \gamma(W_y \vec{h}(x, y, t) {+} \vec{b}_y(x, y, t)) \right|, 
$$
where $W_y {\in} \mathbb{R}^{G {\times} G}$ is a weight matrix and
$\vec{b}_y {\in} \mathbb{R}^G$ is a bias vector. The absolute value is taken so 
to make sure the computed value acts as an increment for either $u_j(x, y, t)$ 
or $d_j(x, y, t)$. 

Biologically speaking $W_{x,ij}, i {\in} [1, G], j {\in} [1, C]$ is whether 
carcinogen $i$ influences gene $j$, $W_{x,iC {+} 1}, i {\in} [1, G]$ is whether 
age of the cell influences gene $j$, $W_{y,ij}$ is whether gene $i$ influences 
gene $j$, and $b_{y,i}(x, y, t)$ is whether gene $i$ has a higher chance of gene 
expression changes relative to other genes.

Let $Z(0, 1)$ be some random variable. The gene expression, $e_j(x, y, t)$, of a 
gene is updated by
\begin{align*}
e_j(x, y, t) &{=} \begin{cases}
                   e_j(x, y, t {-} 1) + \delta_j(x, y, t), &z {\le} \kappa \\
                   e_j(x, y, t {-} 1) - \delta_j(x, y, t), &\text{otherwise}
                 \end{cases} \\
             &{=} \begin{cases}
                   u_j(x, y, t {-} 1) - d_j(x, y, t {-} 1) + \delta_j(x, y, t), &z {\le} \kappa \\
                   u_j(x, y, t {-} 1) - d_j(x, y, t {-} 1) - \delta_j(x, y, t), &\text{otherwise}
                 \end{cases},\\
z {\in} Z, \beta {\in} \mathbb{R}(0, 1).
\end{align*}
A gene $j$ is considered to be mutated if its' gene expression, $e_j(x, y, t)$, 
is above a certain threshold value, which is represented by $M {\in} 
\mathbb{R}_+$. The bias for a gene $j$, $\vec{b}_{y, j}(x, y, t)$, is updated 
through the relation
$$
b_{y, j}(x, y, t) {=} \begin{cases}
                        \beta, &e_j(x, y, t {-} 1) {\ge} M \\
                        0, &\text{otherwise}
                      \end{cases},
\beta {\in} \mathbb{R}_+.
$$

\end{document}