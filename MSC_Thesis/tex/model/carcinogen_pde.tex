\providecommand{\main}{../..}
\documentclass[\main/thesis.tex]{subfiles}

\begin{document}
	
\section{Carcinogen Partial Differential Equations}

Consider $N$ carcinogens which are in the spatial domain \newline
$\Omega {=} \{ (x, y) | 0 {<} x {<} L, 0 {<} y {<} L \}$ and evolve in the 
domain $\Gamma {=} \Omega {\times} (0, \infty)$. \newline
A concentration for each carcinogen is computed by the function \newline
$c_i(x, y, t), i {=} 1, ..., C$. In which $c_i(x, y, t)$ is a solution to the 
following initial boundary value problem (IBVP)
\begin{align}
	&\text{PDE }    \hspace{0.12in} c_i(x, y, t)_t {=} D_i \Delta c_i(x, y, t)
	                                               {+} F_i(x, y, t),
	                                (x, y, t) {\in} \Gamma;
	                                \label{eq:carcin_pde_inhomo} \\
	&\text{BCs }    \hspace{0.17in} c_i(0, y, t)   {=} A_{i,1}(y, t),
	                                c_i(N, y, t)   {=} B_{i,1}(y, t),
	                                y {\in} \partial \Omega, t {>} 0;
	                                \label{eq:bcs_inhomo_x} \\
	&\text{BCs }    \hspace{0.17in} c_i(x, 0, t)   {=} A_{i,2}(x, t),
	                                c_i(x, N, t)   {=} B_{i,2}(x, t),
	                                x {\in} \partial \Omega, t {>} 0;
	                                \label{eq:bcs_inhomo_y} \\
	&\text{IC }     \hspace{0.30in} c_i(x, y, 0)   {=} f_i(x, y),
	                                (x, y) {\in} \Omega;
	                                \label{eq:ic_inhomo} \\
	&\text{Source }                 F_i(x, y, t)   {=} I_i(x, y, t)
	                                               {-} O_i(x, y, t),
	                                \label{eq:source_term}
\end{align}
where
$
\Delta {=} \frac{\partial^2}
                {\partial x^2}
       {+} \frac{\partial^2}
                {\partial y^2}
$;
$F_i(x, y, t) {\in} \mathbb{R}$ is the source term with
$I_i(x, y, t) {\in} \mathbb{R}_+$ being the input and
$O_i(x, y, t) {\in} \mathbb{R}_+$ being the loss of the carcinogen; \newline
$
A_{i,1}(y, t), B_{i,1}(y, t),
A_{i,2}(x, t), B_{i,2}(x, t),
f_i(x, y) {\in} \mathbb{R}_+
$. 

The IBVP \eqref{eq:carcin_pde_inhomo}-\eqref{eq:ic_inhomo} is non-homogeneous 
both in the boundary conditions (BC) and PDE. We can construct a homogeneous 
boundary value problem by introducing the function
\begin{equation*}
	v_i(x, y, t) {:=} c_i(x, y, t) {-} r_i(x, y, t),
\end{equation*}
where $r_i(x, y, t)$ is such that it only satisfies the BC's 
\eqref{eq:bcs_inhomo_x} and \eqref{eq:bcs_inhomo_y}. This function is created so 
that the linear combination of $c_i(x, y, t)$ and $r_i(x, y, t)$ result in 
$v_i(x, y, t)$ satisfying homogeneous versions of \eqref{eq:bcs_inhomo_x} and 
\eqref{eq:bcs_inhomo_y}. Additionally $r_i(x, y, t)$ must be a simple function.
Thus we define the function
\begin{align}
	&r_i(x, y, t)    {=}   H(y {-} x)           A_{i,1}(y, t)
	                 {+}   H(x {-} y)           A_{i,2}(x, t)
	 \nonumber\\
	&\hspace{0.75in} {+} 2 H(x {-} N)          (B_{i,1}(y, t)
	                                        {-} A_{i,2}(x, t))
	 \nonumber\\
	&\hspace{0.75in} {+} 2 H(y {-} N)          (B_{i,2}(x, t)
	                                        {-} A_{i,1}(y, t))
	 \nonumber\\
	&\hspace{0.75in} {+}   H(y {-} x {-} N)    (A_{i,1}(y, t)
	                                        {-} B_{i,2}(x, t)),
	 \nonumber\\
	&\hspace{0.75in} {+}   H(xy {-} N^2)       (A_{i,1}(y, t) {+} A_{i,2}(x, t)
	                                        {-} B_{i,1}(y, t) {-} B_{i,2}(x, t))
	 \nonumber\\
	&\hspace{0.75in} {+}   H(x {-} y {-} N)    (A_{i,2}(x, t) {-} B_{i,1}(y, t)) 
	 \label{eq:ref_carcin_concen}\\
	&H(\xi) {=} \begin{cases}
                    1,           &\xi {>} 0 \\
                    \frac{1}
                         {2},    &\xi {=} 0 \\
                    0,           &\xi {<} 0
                \end{cases}.
	 \label{eq:heaviside}
\end{align}
Insertion of $v_i(x, y, t)$ into 
\eqref{eq:carcin_pde_inhomo}-\eqref{eq:ic_inhomo} results in the IBVP
\begin{align}
	&\text{PDE }    \hspace{0.50in} v_i(x, y, t)_t          {=}  D_i \Delta 
	                                                             v_i(x, y, t)
	                                                        {+} \overline{F}_i(x, y, t), 
	                                (x, y, t) {\in} \Gamma;
	                                \label{eq:carcin_pde_inhomo_v} \\
	&\text{BCs }    \hspace{0.55in} v_i(0, y, t)            {=}  0,
	                                v_i(N, y, t)            {=}  0,
	                                y {\in} \partial \Omega, t {>} 0;
	                                \label{eq:bcs_homo_x} \\
	&\text{BCs }    \hspace{0.55in} v_i(x, 0, t)            {=}  0,
	                                v_i(x, N, t)            {=}  0,
	                                x {\in} \partial \Omega, t {>} 0;
	                                \label{eq:bcs_homo_y} \\
	&\text{IC }     \hspace{0.68in} v_i(x, y, 0)            {=}  g_i(x, y),
	                                \label{eq:ic_inhomo_v} \\
	               &\hspace{0.92in} g_i(x, y)               {:=} f_i(x, y)
	                                                        {-}  r_i(x, y, 0) 
	                                (x, y) {\in} \Omega;
	                                \label{eq:ic_func_inhomo_v} \\
	&\text{Source } \hspace{0.37in} \overline{F}_i(x, y, t) {=}  F_i(x, y, t)
	                                                        {-}  r_i(x, y, t)_t 
	                                                        {+}  D_i \Delta
	                                                             r_i(x, y, t).
	                                \label{eq:source_term_v}
\end{align}

Let's solve the spatial component of the homogeneous boundary value problem 
(BVP) associated with the IBVP 
\eqref{eq:carcin_pde_inhomo_v}-\eqref{eq:ic_inhomo_v} using separation of 
variables and then insert it into \eqref{eq:carcin_pde_inhomo_v} and 
\eqref{eq:ic_inhomo_v} to find a solution for the time dependent function.
The homogeneous BVP associated to 
\eqref{eq:carcin_pde_inhomo_v}-\eqref{eq:ic_inhomo_v} is given by
\begin{align}
	&\text{PDE }                 v_i(x, y, t)_t {=} D_i \Delta v_i(x, y, t),
	                             (x, y, t) {\in} \Gamma;
	                             \label{eq:carcin_pde_homo_v} \\
	&\text{BCs } \hspace{0.05in} v_i(0, y, t)   {=} 0,
	                             v_i(N, y, t)   {=} 0,
	                             y {\in} \partial \Omega, t {>} 0; 
	                             \tag{\ref{eq:bcs_homo_x}} \\
	&\text{BCs } \hspace{0.05in} v_i(x, 0, t)   {=} 0,
	                             v_i(x, N, t)   {=} 0,
	                             x {\in} \partial \Omega t {>} 0. 
	                             \tag{\ref{eq:bcs_homo_y}} 
\end{align}
Assume by separation of variables that 
\begin{equation*}
	v_i(x, y, t) {:=} \phi_i(x, y) \eta_i(t)
\end{equation*}
and insert into \eqref{eq:carcin_pde_homo_v} to acquire
\begin{equation*}
	\phi_i(x, y) \eta_i'(t) {=} D_i \Delta \phi_i(x, y) \eta_i(t).
\end{equation*}
Divide the above by $D_i \phi_i(x, y) \eta_i(t)$ to obtain
\begin{equation*}
	    \frac{\eta_i'(t)}
	         {D_i\eta_i(t)}
	{=} \frac{\Delta \phi_i(x, y)}
	         {\phi_i(x, y)}.
\end{equation*}
Since the left hand side (LHS) of the above only depends on $t$ and the right 
hand side (RHS) only on ($x$, $y$) then set each side equal to some separation 
constant, $\minus \lambda_i$.
This results in the differential equations
\begin{align}
	&\eta_i'(t) {+} \lambda_i D_i \eta_i(t) {=} 0,
	 \nonumber \\
	&\Delta \phi_i(x, y) {+} \lambda_i \phi_i(x, y) {=} 0.
	 \label{eq:spatial_pde}
\end{align}
Taking under consideration the assumed form of $v_i(x, y, t)$ and
$\eta_i(t) {\ne} 0$, the BC's \eqref{eq:bcs_homo_x} and \eqref{eq:bcs_homo_y} 
become
\begin{align}
	\phi_i(0, y) &{=} 0,
	\phi_i(N, y)  {=} 0,
	\label{eq:bcs_homo_phi_x} \\
	\phi_i(x, 0) &{=} 0,
	\phi_i(x, N)  {=} 0.
	\label{eq:bcs_homo_phi_y}
\end{align}
The equations \eqref{eq:spatial_pde}-\eqref{eq:bcs_homo_phi_y} form a BVP which 
can be solved using separation of variables. Assume
\begin{equation*}
	\phi_i(x, y) {:=} \chi_i(x) \psi_i(y)
\end{equation*}
and insert into 
\eqref{eq:spatial_pde} to acquire
\begin{equation*}
	\chi_i''(x) \psi_i(y) {+} \chi_i(x) \psi_i''(y)
	{+} \lambda_i \chi_i(x) \psi_i(y) {=} 0.
\end{equation*} 
Subtracting the above by the second and third terms of the LHS and then dividing 
the result by $\chi_i(x) \psi_i(y)$ yields
\begin{equation*}
	    \frac{\chi_i''(x)}
	         {\chi_i(x)}
	{=} {-} \left(
	              \frac{\psi_i''(y)}
	                   {\psi_i(y)}
	          {+} \lambda_i
	        \right).
\end{equation*}
The LHS of the previous equation depends only on $x$ while the RHS only on $y$. 
Thus setting both sides equal to a separation constant, $\minus \mu_i$, acquires 
the following ordinary differential equations (ODE)
\begin{align}
	&\chi_i''(x) {+} \mu_i \chi_i(x) {=} 0,
	\label{eq:ode_x} \\
	&\psi_i''(y) {+} (\lambda_i {-} \mu_i) \psi_i(y) {=} 0.
	\label{eq:ode_y}
\end{align}
Insert $\phi_i(x, y)$ into the BC's \eqref{eq:bcs_homo_phi_x} and 
\eqref{eq:bcs_homo_phi_y}, with the assumption that the trivial solutions of the 
ODE's are undesireable, to acquire
\begin{align}
	&\chi_i(0) {=} 0, \chi_i(N) {=} 0,
	\label{eq:bc_ode_x} \\
	&\psi_i(0) {=} 0, \psi_i(N) {=} 0.
	\label{eq:bc_ode_y}
\end{align} 
Two Sturm-Liouville problems (SLP), namely (\ref{eq:ode_x}, \ref{eq:bc_ode_x}) 
and (\ref{eq:ode_y}, \ref{eq:bc_ode_y}) have resulted from the separation of 
variables. Since (\ref{eq:ode_y}, \ref{eq:bc_ode_y}) depends on two separation 
constants, ($\lambda_i$, $\mu_i$), and (\ref{eq:ode_x}, \ref{eq:bc_ode_x}) only 
on the constant $\mu_i$ then (\ref{eq:ode_x}, \ref{eq:bc_ode_x}) must be solved 
first. The equation \eqref{eq:ode_x} is a homogeneous second order linear ODE 
with constant coefficients in the canonical form: 
$au''(\xi) {+} bu'(\xi) {+} du(\xi) {=} 0$. Therefore the solution can be 
classified using the discriminant of the characteristic equation
$a\nu^2 {+} b\nu {+} d {=} 0$, whereby, if it is:
\begin{align*}
	&\text{strictly positive then } \hspace{0.03in} u(\xi) {=} K_1 \cosh(\nu_1 \xi) 
	                                                       {+} K_2 \sinh(\nu_2 \xi),
	 \\
	&\text{strictly negative then }                 u(\xi) {=} e^{\text{Re}(\nu) \xi}
	                                                           \left(
	                                                                 K_1 \cos(\text{Im}(\nu) \xi)
	                                                             {+} K_2 \sin(\text{Im}(\nu) \xi)
	                                                           \right),
	 \\
	&\text{zero then }              \hspace{0.87in} u(\xi) {=} K_1 e^{\nu \xi} 
	                                                       {+} K_2 \xi e^{\nu \xi}.
\end{align*}
For \eqref{eq:ode_x} $a {=} 1, b {=} 0, d {=} \mu_i$ results in a strictly 
negatively discriminant and $\nu {=} {\pm}\sqrt{\mu_i}i, \mu {>} 0$. Thus the 
general solution is given by
\begin{equation*}
	\chi_i(x) {=} K_{i,1} \cos(\sqrt{\mu_i} x)
	          {+} K_{i,2} \sin(\sqrt{\mu_i} x), 
	K_{i,1}, K_{i,2} {\in} \mathbb{R}.
\end{equation*}
The first BC in \eqref{eq:bc_ode_x} implies that $K_{i,1} {=} 0$, so that
\begin{equation*}
	\chi_i(x) {=} K_{i,2} \sin(\sqrt{\mu_i} x).
\end{equation*}
The second BC in \eqref{eq:bc_ode_x} results in $K_{i,2} \sin(\sqrt{\mu_i} N) {=} 0$
which implies that either $K_{i,2} {=} 0$ or $\sin(\sqrt{\mu_i} N) {=} 0$. 
Choose $K_{i,2} {=} 1$ and set $\sin(\sqrt{\mu_i} N) {=} 0$, so that $\chi_i(x) {\ne} 0$,
to arrive at the solution
\begin{align*}
	&\chi_{i,n}(x) {=} \sin(\sqrt{\mu_{i,n}} x), \\
	&\mu_{i,n}     {=} \frac{(n {+} 1)^2 \pi^2}
	                        {N^2},
	 n {=} 0, 1, 2, ....
\end{align*}
Note that the SLP (\ref{eq:ode_y}, \ref{eq:bc_ode_y}) differs from the SLP 
(\ref{eq:ode_x}, \ref{eq:bc_ode_x}) only in that the characteristic equation has 
$d {=} \lambda_i {-} \mu_{i,n}$ and thus the solution is
\begin{align*}
	&\psi_{i,m}(y)   {=} \sin(\sqrt{\lambda_{i,nm} {-} \mu_{i,m}} y), \\
	&\lambda_{i,nm} {=} \frac{((n {+} 1)^2 {+} (m {+} 1)^2)) \pi^2}
	                          {N^2},
	 n, m {=} 0, 1, 2, ....
\end{align*}
Therefore the solution of the BVP 
\eqref{eq:spatial_pde}-\eqref{eq:bcs_homo_phi_y} is given by
\begin{align}
	&\phi_{i,nm}(x, y) {=} \sin\left( 
	                              \frac{(n {+} 1) \pi x}
	                                   {N} 
	                            \right)
	                        \sin\left(
	                              \frac{(m {+} 1) \pi y}
	                                   {N}
	                            \right),
	 \label{eq:spatial_sol} \\
	&\lambda_{i,nm}    {=} \frac{((n {+} 1)^2 {+} (m {+} 1)^2) \pi^2}
	                            {N^2},
	 n,m {=} 0, 1, 2, ....
	 \label{eq:sep_const}
\end{align}
Thus the solution of the BVP (\ref{eq:carcin_pde_homo_v}, \ref{eq:bcs_homo_x}, 
\ref{eq:bcs_homo_y}) is
\begin{equation*}
	v_{i,nm}(x, y, t) {=} \phi_{i,nm}(x, y) \eta_{i,nm}(t).
\end{equation*}
Since $v_{i,nm}(x, y, t)$ is a set of infinite solutions and a superposition of 
solutions is also a solution then $v_{i,nm}(x, y, t)$ can be written as an 
infinite sum, \ie,
\begin{equation}
	v_i(x, y, t) {=} \sum_{n,m {=} 0}^{\infty} \phi_{i,nm}(x, y) \eta_{i,nm}(t).
	\label{eq:variable_sol}
\end{equation}
Insert this infinite sum into \eqref{eq:carcin_pde_inhomo_v} to obtain 
\begin{equation*}
	    \sum_{n,m {=} 0}^{\infty} \phi_{i,nm}(x, y) \eta'_{i,nm}(t)
	{=} D_i \sum_{n,m {=} 0}^{\infty} \Delta \phi_{i,nm}(x, y) \eta_{i,nm}(t) 
	{+} \overline{F}_i(x, y, t). 
\end{equation*}
By \eqref{eq:spatial_pde} we know
$\Delta \phi_{i,nm}(x, y) {=} \minus \lambda_{i,nm} \phi_{i,nm}(x, y)$ and therefore
\begin{equation*}
	    \sum_{n,m {=} 0}^{\infty} \eta'_{i,nm}(t) \phi_{i,nm}(x, y) 
	{=} {-} \sum_{n,m {=} 0}^{\infty} \lambda_{i,nm} D_i \eta_{i,nm}(t) 
	                                  \phi_{i,nm}(x, y) 
	{+} \overline{F}_i(x, y, t).
\end{equation*}
Adding the first term of the RHS obtains
\begin{equation}
	    \sum_{n,m {=} 0}^{\infty}
	      \left( 
	        \eta'_{i,nm}(t) {+} \lambda_{i,nm} D_i \eta_{i,nm}(t)
	      \right) 
	      \phi_{i,nm}(x, y)
	{=} \overline{F}_i(x, y, t).
	\label{eq:time_source_relation}
\end{equation} 
Since any function can be represented by a generalized Fourier series then define
\begin{equation*}
	\overline{F}_{i}(x, y, t) {:=} \sum_{n,m {=} 0}^{\infty} h_{i,nm}(t) 
	                                                         \phi_{i,nm}(x, y),
\end{equation*}
where due to Fourier analysis the Fourier coefficient $h_{i,nm}(t)$ are
\begin{equation*}
	h_{i,nm}(t) {=} \frac{\int_{0}^{N} \int_{0}^{N} \overline{F}_i(x, y, t) \phi_{i,nm}(x, y) dydx}
	                     {\int_{0}^{N} \int_{0}^{N} \phi_{i,nm}^2(x, y) dydx}.
\end{equation*}
Inserting $\phi_{i,nm}(x, y)$ into the denominator results with
\begin{equation*}
	h_{i,nm}(t) {=} \frac{\int_{0}^{N} \int_{0}^{N} \overline{F}_i(x, y, t)	\phi_{i,nm}(x, y) dydx}
	                     {\int_{0}^{N} \sin^2\left( \frac{(n {+} 1) \pi x}{N} \right) dx
	                      \int_{0}^{N} \sin^2\left( \frac{(m {+} 1) \pi y}{N} \right) dy}.
\end{equation*}
The integrals in the denominator have the same integrand albeit in different 
variables, thus only one of them needs explicit computation and the other 
follows immediately. Choosing to compute the integral that is \wrt to $x$ and 
using $\cos(2 \theta) {=} 1 {-} 2 \sin^2(\theta)$ obtains 
\begin{equation*}
	\frac{1}
	     {2}
	\int_{0}^{N} 1 {-} \cos\left( 
	                         \frac{2 (n {+} 1) \pi x}
	                              {N}
	                       \right) dx.
\end{equation*} 
Computing the above integral and taking into consideration that \newline
$\sin(\xi \pi) {=} 0, \xi {\in} \mathbb{Z}$ results with
\begin{equation*}
	    \int_{0}^{N} \sin^2\left( 
	                         \frac{(n {+} 1) \pi x}
	                              {N}
	                       \right) dx
	{=} \frac{N}
	         {2}.
\end{equation*}
Therefore, it follows that
\begin{equation*}
	    \int_{0}^{N} \sin^2\left( 
	                         \frac{(m {+} 1) \pi y}
	                              {N}
	                       \right) dy
    {=} \frac{N}
             {2}.
\end{equation*}
Thus the Fourier coefficient for $\overline{F}_{i}(x, y, t)$ is 
\begin{equation}
	h_{i,nm}(t) {=} \frac{4}
	                     {N^2}
	                \int_{0}^{N} \int_{0}^{N} \overline{F}_i(x, y, t)
	                                          \phi_{nm}(x, y) dydx. 
	\label{eq:fourier_coeff_F}
\end{equation} 
Under the new definition of $\overline{F}_{i}(x, y, t)$ 
\eqref{eq:time_source_relation} becomes
\begin{equation*}
	    \sum_{n,m {=} 0}^{\infty}
	      \left(
	            \eta'_{i,nm}(t)
	        {+} \lambda_{i,nm} D_i \eta_{i,nm}(t)
	        {-} h_{i,nm}(t) 
	      \right) 
	      \phi_{i,nm}(x, y)
    {=} 0.
\end{equation*}
The above is an orthogonal relationship which implies that since
$\phi_{i,nm}(x, y) {\ne} 0$ and $\eta_{i,nm}(t)$ is to be solved that it must be 
that
\begin{equation*}
	\eta'_{i,nm}(t) {+} \lambda_{i,nm} D_i \eta_{i,nm}(t) {=} h_{i,nm}(t).
\end{equation*}
This equation is a first order linear ODE which can be solved by multiplying it 
by an integrating factor, so that the LHS is the result of an application of the 
product rule and thus can be written as a derivate of the product between the 
dependent term, $\eta_{i,nm}(t)$, and the integrating factor, after which it can 
be integrated \wrt the independent variable, $t$. Using this technique with the 
integrating factor $e^{\lambda_{i,nm} D_i t}$ obtains  
\begin{equation*}
	    e^{\lambda_{i,nm} D_i t} \eta_{i,nm}(t)
	{=}
	    \int_{0}^{t} e^{\lambda_{i,nm} D_i \tau} h_{i,nm}(\tau) d\tau
	{+} K_{i,nm}, 
	K_{i,nm} {\in} \mathbb{R}.
\end{equation*}
Multiplying the above by $e^{\minus \lambda_{i,nm} D_i t}$ results with the general solution
\begin{equation}
	\eta_{i,nm}(t) {=} e^{\minus \lambda_{i,nm} D_i t}
	                   \left( 
	                         \int_{0}^{t} e^{\lambda_{i,nm} D_i \tau} 
	                                      h_{i,nm}(\tau) d\tau
	                     {+} K_{i,nm}
	                   \right),
	K_{i,nm} {\in} \mathbb{R}
	\label{eq:time_depen_sol}
\end{equation}
Applying the initial condition (IC) \eqref{eq:ic_inhomo_v} to acquire a 
formulation of $K_{i,nm}$ which in turn obtains the particular solution, results in
\begin{equation*}
	v_{i}(x, y, 0) {=} \sum_{n,m {=} 0}^{\infty} \eta_{i,nm}(0) \phi_{i,nm}(x, y)
	               {=} g_i(x, y).
\end{equation*}
By the rules of exponents $e^0 {=} 1$ and $\int_{0}^{0} \gamma(\xi) d\xi {=} 0$ 
it is easily shown that \newline
$\eta_{i,nm}(0) {=} K_{i,nm}$ and therefore
\begin{equation*}
	g_{i}(x, y) {=} \sum_{n,m {=} 0}^{\infty} K_{i,nm} \phi_{i,nm}(x, y).
\end{equation*}
Therefore, $K_{i,nm}$ is the Fourier coefficient of the generalized Fourier 
series of $g_i(x, y)$ and thus
\begin{equation*}
	K_{i,nm} {=} \frac{\int_{0}^{N} \int_{0}^{N} g_i(x, y) \phi_{i,nm}(x, y) dydx}
	                  {\int_{0}^{N} \int_{0}^{N} \phi_{i,nm}^2(x, y) dydx}.
\end{equation*}
Similarly to $h_{i,nm}(t)$ the denominator is $\frac{N^2}{4}$ and as a result
\begin{equation}
	K_{i,nm} {=} \frac{4}
	                  {N^2}
	             \int_{0}^{N}\int_{0}^{N} g_i(x, y) \phi_{i,nm}(x, y) dydx.
	\label{eq:fourier_coeff_g}
\end{equation}
Accumulating the above results gives the solution to the IBVP 
\eqref{eq:carcin_pde_inhomo}-\eqref{eq:ic_inhomo}  
\begin{align}
	                &c_i(x, y, t)            {=} v_{i}(x, y, t) {+} r_i(x, y, t),
	                 \label{eq:carcin_sol} \\
	&\hspace{0.25in} v_{i}(x, y, t)          {=} \sum_{n,m {=} 0}^{\infty}
	                                               \eta_{i,nm}(t) \phi_{i,nm}(x, y), 
	                 \tag{\ref{eq:variable_sol}} \\
	&\hspace{0.50in} \eta_{i,nm}(t)          {=} e^{\minus \lambda_{i,nm} D_i t}
	                                             \left(
	                                               \int_{0}^{t} 
	                                                 e^{\lambda_{i,nm} D_i \tau}
	                                                 h_{i,nm}(\tau)
	                                               d\tau
	                                               {+} K_{i,nm} 
	                                             \right),
	                 \tag{\ref{eq:time_depen_sol}} \\
	&\hspace{0.75in} \lambda_{i,nm}          {=} \frac{((n {+} 1)^2 {+} (m {+} 1)^2) \pi^2}
	                                                  {N^2},
	                 n, m {=} 0, 1, 2, ...,
	                 \tag{\ref{eq:sep_const}} \\
	&\hspace{0.75in} h_{i,nm}(t)             {=} \frac{4}
	                                                  {N^2}
	                                             \int_{0}^{N} \int_{0}^{N} 
	                                               \overline{F}_i(x, y, t) 
	                                               \phi_{i,nm}(x, y)
	                                             dydx,
	                 \tag{\ref{eq:fourier_coeff_F}} \\
	&\hspace{1.00in} \overline{F}_i(x, y, t) {=} F_i(x, y, t) {-} r_i(x, y, t)_t
	                                         {+} D_i \Delta r_i(x, y, t), 
	                 \tag{\ref{eq:source_term_v}} \\
	&\hspace{0.75in} K_{i,nm}                {=} \frac{4}
	                                                  {N^2}
	                                             \int_{0}^{N} \int_{0}^{N} 
	                                               g_i(x, y) \phi_{i,nm}(x, y)
	                                             dydx, 
	                 \tag{\ref{eq:fourier_coeff_g}} \\
	&\hspace{1.00in} g_i(x, y)               {=} f_i(x, y) {-} r_i(x, y, 0), 
	                 \tag{\ref{eq:ic_func_inhomo_v}} \\
	&\hspace{0.50in} \phi_{i,nm}(x, y)       {=} \sin\left(
	                                                   \frac{(n {+} 1) \pi x}
	                                                        {N} 
	                                                 \right) 
	                                             \sin\left(
	                                                   \frac{(m {+} 1) \pi y}
	                                                        {N}
	                                                 \right),
	                 \tag{\ref{eq:spatial_sol}}\\
	&\hspace{0.25in} r_i(x, y, t)            {=}   H(y {-} x)          	A_{i,1}(y, t)
	                                         {+}   H(x {-} y)           A_{i,2}(x, t)
	                 \nonumber \\
	&\hspace{1.00in}                         {+} 2 H(x {-} N)          (B_{i,1}(y, t) 
	                                                                {-} A_{i,2}(x, t))
	                 \nonumber \\
	&\hspace{1.00in}                         {+} 2 H(y {-} N)      	   (B_{i,2}(x,t) 
	                                                                {-} A_{i,1}(y, t))
	                 \nonumber \\
	&\hspace{1.00in}                         {+}   H(y {-} x {-} N)    (A_{i,1}(y, t)
	                                                                {-} B_{i,2}(x, t)),
	                 \nonumber \\
	&\hspace{1.00in}                         {+}   H(xy {-} N^2)       (A_{i,1}(y, t)
	                                                                {+} A_{i,2}(x, t) 
	                                                                {-} B_{i,1}(y, t) 
	                                                                {-} B_{i,2}(x, t))
	                 \nonumber \\
	&\hspace{1.00in}                         {+}   H(x {-} y {-} N)    (A_{i,2}(x, t) 
                           	                                        {-} B_{i,1}(y, t))
	                 \tag{\ref{eq:ref_carcin_concen}} \\
	&\hspace{0.50in} H(\xi) {=} \begin{cases}
	                                1,           &\xi {>} 0 \\
		                            \frac{1}
		                                 {2},    &\xi {=} 0 \\
		                            0,           &\xi {<} 0
	                            \end{cases}.
	\tag{\ref{eq:heaviside}}\\
\end{align}

\end{document}