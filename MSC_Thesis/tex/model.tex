\providecommand{\main}{..}
\documentclass[\main/thesis.tex]{subfiles}

\begin{document}

\chapter{Model}

\section{Model Overview}
The hybrid cellular automaton (CA) model proposed is one that simulates field cancerization. It is a hybrid model because the CA rule depends on the output of other mathematical objects. 
In this case the two mathematical objects are partial differential equations (PDE) and neural networks (NN). The PDE models spread of one or more carcinogen(s) within the domain of the CA.
A NN is used to compute the change in gene expression of the genes under consideration for each cell with respect to the amount of carcinogen at the cell's location and it's age. 
The CA has states that are used to represent the following biological cell types: normal tissue cells (NTC), mutated normal tissue cells (MNTC), normal stem cells (NSC), mutated normal stem cells (MNSC), 
cancer stem cells (CSC), and tumour cells (TC).
Evolution of the model occurs in the following basic steps:
\begin{enumerate}
	\item Carcinogens are allowed to spread via the PDE.
	\item Changes in gene expressions for the genes are computed by the NN and gene mutations are allowed to occur. 
	\item All the cells are updated using the CA rule. 
\end{enumerate}

\section{Carcinogen Partial Differential Equations}
We consider $C$ carcinogen's which are in the spatial domain \newline
$\Omega {=} \{ (x, y) | 0 {<} x {<} N, 0 {<} y {<} N \}$ and evolve in the domain $\Omega_t {=} \Omega {\times} (0, \infty)$.\newline
A concentration for each carcinogen is computed by the function \newline
$c_i(x, y, t), i=1, ..., C$. $c_i(x, y, t)$ is a solution to the following initial boundary value problem (IBVP)
\begin{align}
\text{PDE } c_i(x, y, t)_t &{=} D_i\Delta c_i(x, y, t) {+} F_i(x, y, t), (x, y, t) {\in} \Omega_t,
\label{eq:carcin_pde_inhomo} \\
\text{BCs } c_i(0, y, t) &{=} A_{i, 1}(y, t), c_i(N, y, t) {=} B_{i, 1}(y, t), (y, t) {\in} (0, N) {\times} (0, \infty),
\label{eq:bcs_inhomo_x} \\
\text{BCs } c_i(x, 0, t) &{=} A_{i, 2}(x, t), c_i(x, N, t) {=} B_{i, 2}(x, t), (x, t) {\in} (0, N) {\times} (0, \infty),
\label{eq:bcs_inhomo_y} \\
\text{IC } c_i(x, y, 0) &{=} f_i(x, y), (x, y) {\in} \Omega,
\label{eq:ic_inhomo} \\
\text{Source } F_i(x, y, t) &{=} I_i(x, y, t) {-} O_i(x, y, t),
\label{eq:source_term}
\end{align}
where $\Delta {=} \frac{\partial^2}{\partial x^2} + \frac{\partial^2}{\partial y^2}$; $F_i(x, y, t) {\in} \mathbb{R}$ is the source term with $I_i(x, y, t) {\in} \mathbb{R}_+$ being the influx and 
$O_i(x, y, t) {\in} \mathbb{R}_+$ being the outflux of the carcinogen; \\ $A_{i, 1}(y, t), B_{i, 1}(y, t), A_{i, 2}(x, t), B_{i, 2}(x, t), f_i(x, y) {\in} \mathbb{R}_+$.  

The IBVP \eqref{eq:carcin_pde_inhomo}-\eqref{eq:ic_inhomo} is non-homogeneous both in the boundary conditions (BC) and PDE. As a result we will try to simplify the problem by formulating a new IBVP that 
has homogeneous boundary conditions. We introduce the function: $v_i(x, y, t) {:=} c_i(x, y, t) {-} r_i(x, y, t)$, where $r_i(x, y, t)$ is such that it only satisfies the BC's \eqref{eq:bcs_inhomo_x} 
and \eqref{eq:bcs_inhomo_y}. This function is created so that the linear combination of $c_i(x, y, t)$ and $r_i(x, y, t)$ result in $v_i(x, y, t)$ satisfying homogeneous versions of 
\eqref{eq:bcs_inhomo_x} and \eqref{eq:bcs_inhomo_y}. Additionally we want $r_i(x, y, t)$ to be a simple function. Thus we define the function
\begin{align}
r_i(x, y, t) {=} &H(y{-}x)A_{i, 1}(y, t) {+} H(x{-}y)A_{i, 2}(x, t) \nonumber\\
				 &{+} 2H(x{-}N)(B_{i, 1}(y, t) {-} A_{i, 2}(x, t)) {+} 2H(y{-}N)(B_{i, 2}(x, t) {-} A_{i, 1}(y, t)) \nonumber\\
				 &{+} H(y{-}x{-}N)(A_{i, 1}(y, t) {-} B_{i, 2}(x, t)), \label{eq:ref_carcin_concen}\\
                 &{+} H(xy{-}N^2)(A_{i, 1}(y, t) {+} A_{i, 2}(x, t) {-} B_{i, 1}(y, t) {-} B_{i, 2}(x, t)) \nonumber\\
                 &{+} H(x{-}y{-}N)(A_{i, 2}(x, t) {-} B_{i, 1}(y, t)) \nonumber\\
H(\xi) {=} &\begin{cases}
	1, &\xi {>} 0 \\
	\frac{1}{2}, &\xi {=} 0 \\
	0, &\xi {<} 0
\end{cases}.
\label{eq:heaviside}
\end{align}
Upon plugging $v_i(x, y, t)$ into \eqref{eq:carcin_pde_inhomo}-\eqref{eq:ic_inhomo} we get the IBVP
\begin{align}
\text{PDE } v_i(x, y, t)_t &{=} D_i\Delta v_i(x, y, t) {+} \overline{F}_i(x, y, t), (x, y, t) {\in} \Omega_t,
\label{eq:carcin_pde_inhomo_v} \\
\text{BCs } v_i(0, y, t) &{=} 0, v_i(N, y, t) {=} 0, (y, t) {\in} (0, N) {\times} (0, \infty),
\label{eq:bcs_homo_x} \\
\text{BCs } v_i(x, 0, t) &{=} 0, v_i(x, N, t) {=} 0, (x, t) {\in} (0, N) {\times} (0, \infty),
\label{eq:bcs_homo_y} \\
\text{IC } v_i(x, y, 0) &{=} g(x, y), (x, y) {\in} \Omega,
\label{eq:ic_inhomo_v} \\
\text{Source } \overline{F}_i(x, y, t) &{=} F_i(x, y, t) {-} r_i(x, y, t)_t {+} D_i\Delta r_i(x, y, t),
\label{eq:source_term_v} \\
\text{IC Function } g_i(x, y) &{=} f_i(x, y) {-} r_i(x, y, 0).
\label{eq:ic_func_inhomo_v}
\end{align}

Let's solve the spatial component of the homogeneous boundary value problem (BVP) associated with the IBVP \eqref{eq:carcin_pde_inhomo_v}-\eqref{eq:ic_inhomo_v} using separation of variables and then 
plug it into \eqref{eq:carcin_pde_inhomo_v} and \eqref{eq:ic_inhomo_v} to find a solution for the time dependent part.
The homogeneous BVP associated to \eqref{eq:carcin_pde_inhomo_v}-\eqref{eq:ic_inhomo_v} is given by
\begin{align}
\text{PDE } v_i(x, y, t)_t &{=} D_i \Delta v_i(x, y, t), (x, y, t) {\in} \Omega_t,
\label{eq:carcin_pde_homo_v} \\
\text{BCs } v_i(0, y, t) &{=} 0, v_i(N, y, t) {=} 0, (y, t) {\in} (0, N) {\times} (0, \infty), \tag{\ref{eq:bcs_homo_x}} \\
\text{BCs } v_i(x, 0, t) &{=} 0, v_i(x, N, t) {=} 0, (x, t) {\in} (0, N) {\times} (0, \infty). \tag{\ref{eq:bcs_homo_y}} 
\end{align}
As mentioned before we will solve the BVP \eqref{eq:carcin_pde_homo_v}, \eqref{eq:bcs_homo_x}, and \eqref{eq:bcs_homo_y} by separation of variables. Assume $v_i(x, y, t) {:=} \phi_i(x, y)T_i(t)$ and 
plug it into \eqref{eq:carcin_pde_homo_v} we acquire
\begin{equation*}
\phi_i(x, y)T_i'(t) {=} D_i\Delta \phi_i(x, y)T_i(t).
\end{equation*}
Dividing the above by $D_i\phi_i(x, y)T_i(t)$ we get
\begin{equation*}
\frac{T_i'(t)}{D_iT_i(t)} {=} \frac{\Delta \phi_i(x, y)}{\phi_i(x, y)}.
\end{equation*}
Since the left hand side (LHS) of the above only depends on $t$ and the right hand side (RHS) only on ($x$, $y$) then we can set each side equal to some separation constant, $\minus \lambda_i$.
Thus we have
\begin{align}
T_i'(t) {+} \lambda_i D_i T_i(t) {=} 0, \nonumber \\
\Delta \phi_i(x, y) {+} \lambda_i \phi_i(x, y) {=} 0.
\label{eq:spatial_pde}
\end{align}
Taking under consideration the assumed form of $v_i(x, y, t)$ and that we don't want $T_i(t) {=} 0$, the BC's \eqref{eq:bcs_homo_x} and \eqref{eq:bcs_homo_y} become
\begin{align}
\phi_i(0, y) &{=} 0, \phi_i(N, y) {=} 0,
\label{eq:bcs_homo_phi_x} \\
\phi_i(x, 0) &{=} 0, \phi_i(x, N) {=} 0.
\label{eq:bcs_homo_phi_y}
\end{align}
The equations \eqref{eq:spatial_pde}-\eqref{eq:bcs_homo_phi_y} form a BVP which can be solved using separation of variables. Assuming $\phi_i(x, y) {:=} X_i(x)Y_i(y)$ and plugging it into 
\eqref{eq:spatial_pde} leads to
\begin{equation*}
X_i''(x)Y_i(y) {+} X_i(x)Y_i''(y) {+} \lambda_i X_i(x)Y_i(y) {=} 0.
\end{equation*} 
Subtracting the above by the second and third terms of the LHS and then dividing the result by $X_i(x)Y_i(y)$ yields
\begin{equation*}
\frac{X_i''(x)}{X_i(x)} {=} {-}\left(\frac{Y_i''(y)}{Y_i(y)} {+} \lambda_i \right).
\end{equation*}
One sees that the LHS of the previous equation depends only on $x$ while the RHS only on $y$. Thus setting both sides equal to a separation constant, $\minus \mu_i$, we acquire the following ordinary 
differential equations (ODE)
\begin{align}
X_i''(x) {+} \mu_i X_i(x) {=} 0,
\label{eq:ode_x} \\
Y_i''(y) {+} (\lambda_i {-} \mu_i)Y_i(y) {=} 0.
\label{eq:ode_y}
\end{align}
As a result of not wanting the trivial solutions of the above ODE's and plugging $\phi_i(x, y)$ into the BC's \eqref{eq:bcs_homo_phi_x} and \eqref{eq:bcs_homo_phi_y} we have
\begin{align}
X_i(0) {=} 0, X_i(N) {=} 0,
\label{eq:bc_ode_x}\\
Y_i(0) {=} 0, Y_i(N) {=} 0.
\label{eq:bc_ode_y}
\end{align} 
We have two Sturm-Liouville problems (SLP), namely \eqref{eq:ode_x}, \eqref{eq:bc_ode_x} and \eqref{eq:ode_y}, \eqref{eq:bc_ode_y}. Since \eqref{eq:ode_y}, \eqref{eq:bc_ode_y} depends on two separation 
constants, ($\lambda_i$, $\mu_i$), and \eqref{eq:ode_x}, \eqref{eq:bc_ode_x} only on the constant $\mu_i$ then \eqref{eq:ode_x}, \eqref{eq:bc_ode_x} must be solved first. The equation \eqref{eq:ode_x} is 
a homogeneous second order linear ODE with constant coefficients in the canonical form: $au''(\xi) {+} bu'(\xi) {+} cu(\xi) {=} 0$. Therefore we can classify its solution using the discriminant of the 
characteristic equation $a\nu^2 {+} b\nu {+} c {=} 0$, whereby, if it is:
\begin{align*}
&\text{strictly positive then } &u(\xi) &{=} C_1\cosh(\nu_1\xi) {+} C_2\sinh(\nu_2\xi),\\
&\text{strictly negative then } &u(\xi) &{=} e^{\text{Re}(\nu) \xi}(C_1\cos(\text{Im}(\nu) \xi) {+} C_2\sin(\text{Im}(\nu) \xi)),\\
&\text{zero then } &u(\xi) &{=} C_1 e^{\nu \xi} + C_2\xi e^{\nu \xi}.
\end{align*}
In our case $a {=} 1, b {=} 0, c {=} \mu_i$ and so we have a strictly negatively discriminant and $\nu {=} {\pm}\sqrt{\mu_i}i, \mu {>} 0$. Thus the general solution of \eqref{eq:ode_x} is given by
\begin{equation*}
X_i(x) {=} C_{i,1}\cos(\sqrt{\mu_i} x) + C_{i,2}\sin(\sqrt{\mu_i} x), C_{i,1}, C_{i,2} {\in} \mathbb{R}.
\end{equation*}
Applying the first BC in \eqref{eq:bc_ode_x} to the above we get that $C_{i,1} {=} 0$, so that $X_i(x) {=} C_{i,2}\sin(\sqrt{\mu_i} x)$. Now applying the second BC in \eqref{eq:bc_ode_x} we arrive at 
the relation $C_{i,2}\sin(\sqrt{\mu_i} N) {=} 0$ which implies that either $C_{i,2} {=} 0$ or $\sin(\sqrt{\mu_i} N) {=} 0$. It must be that $\sin(\sqrt{\mu_i} N) {=} 0$ as we don't want $X_i(x) {=} 0$ 
and so \newline
$\mu_{i,n} {=} \frac{(n+1)^2\pi^2}{N^2}, n {=} 0, 1, 2, ...$. Upon choosing $C_{i,2} {=} 1$ we arrive at the solution
\begin{align*}
X_{i,n}(x) &{=} \sin(\sqrt{\mu_{i,n}} x), \\
\mu_{i,n} &{=} \frac{(n+1)^2\pi^2}{N^2}, n {=} 0, 1, 2, ....
\end{align*}
Now we can solve the SLP \eqref{eq:ode_y}, \eqref{eq:ode_y}. It is easily seen that the problem differs from the SLP for $X_i(x)$ only in that the characteristic equation has 
$c {=} \lambda_i {-} \mu_{i, n}$ and so the solution is
\begin{align*}
Y_{i,m}(y) &{=} \sin(\sqrt{\lambda_{i, nm}{-}\mu_{i, m}} y), \\
\lambda_{i, nm} &{=} \frac{((n+1)^2 {+} (m+1)^2))\pi^2}{N^2}, n,m {=} 0, 1, 2, ....
\end{align*}
Therefore the solution of the BVP \eqref{eq:spatial_pde}-\eqref{eq:bcs_homo_phi_y} is given by
\begin{align}
\phi_{i,nm}(x, y) &{=} \sin\left( \frac{(n+1)\pi x}{N} \right) \sin\left( \frac{(m+1)\pi y}{N} \right),
\label{eq:spatial_sol} \\
\lambda_{i, nm} &{=} \frac{((n+1)^2 {+} (m+1)^2)\pi^2}{N^2}, n,m {=} 0, 1, 2, ....
\label{eq:sep_const}
\end{align}
Thus the solution of the BVP \eqref{eq:carcin_pde_homo_v}-\eqref{eq:bcs_homo_y} is $v_{i,nm}(x, y, t) {=} \phi_{i,nm}(x, y)T_{i,nm}(t)$. Since $v_{i,nm}(x, y, t)$ is a set of infinite solutions and a 
superposition of solutions is also a solution then we can write $v_{i,nm}(x, y, t)$ as an infinite sum, \ie,
\begin{equation}
v_{i,nm}(x, y, t) {=} \sum_{n,m {=} 0}^{\infty} \phi_{i,nm}(x, y)T_{i,nm}(t).
\label{eq:variable_sol}
\end{equation}
Inserting this infinite sum into \eqref{eq:carcin_pde_inhomo_v} we get
\begin{equation*}
\sum_{n,m {=} 0}^{\infty} \phi_{i,nm}(x, y)T'_{i,nm}(t) {=} D_i \sum_{n,m {=} 0}^{\infty} \Delta \phi_{i,nm}(x, y) T_{i,nm}(t) {+} \overline{F}_i(x, y, t).
\end{equation*}
By \eqref{eq:spatial_pde} we know $\Delta \phi_{i,nm}(x, y) {=} \minus \lambda_{i,nm} \phi_{i,nm}(x, y)$ so that the above becomes
\begin{equation*}
\sum_{n,m{=}0}^{\infty} T'_{i,nm}(t) \phi_{i,nm}(x, y) = {-}\sum_{n,m{=}0}^{\infty} \lambda_{i,nm} D_i T_{i,nm}(t) \phi_{i,nm}(x, y) {+} \overline{F}_i(x, y, t).
\end{equation*}
Adding the first term of the RHS to the above we get:
\begin{equation}
\sum_{n,m{=}0}^{\infty} \left( T'_{i,nm}(t) {+} \lambda_{i,nm} D_i T_{i,nm}(t) \right)\phi_{i,nm}(x, y) {=} \overline{F}_i(x, y, t).
\label{eq:time_source_relation}
\end{equation} 
Since any function can be represented by a generalized Fourier series then we can define
\begin{equation*}
\overline{F}_{i,nm}(x, y, t) {:=} \sum_{n,m{=}0}^{\infty} h_{i,nm}(t) \phi_{i,nm}(x, y),
\end{equation*}
where from Fourier analysis we know that the Fourier coefficient $h_{i,nm}(t)$ is
\begin{equation*}
h_{i,nm}(t) {=} \frac{\int_{0}^{N} \int_{0}^{N} \overline{F}_i(x, y, t) \phi_{i,nm}(x, y) dy dx}{\int_{0}^{N} \int_{0}^{N} \phi_{i,nm}^2(x, y) dy dx}.
\end{equation*}
Plugging in $\phi_{i,nm}(x, y)$ into the denominator gives us
\begin{equation*}
h_{i,nm}(t) {=}
\frac{\int_{0}^{N} \int_{0}^{N} \overline{F}_i(x, y, t) \phi_{i,nm}(x, y) dy dx}{\int_{0}^{N} \sin^2\left( \frac{(n+1)\pi x}{N} \right) dx \int_{0}^{N} \sin^2\left( \frac{(m+1)\pi y}{N} \right) dy}.
\end{equation*}
Since the two integrals have the same integrand albeit in different variables we need only compute one of them and the other follows immediately. We will choose to explicitly compute the integral that 
is with respect to $x$. Using the trigonometric identity $\cos(2 \theta) {=} 1 {-} 2\sin^2(\theta)$ the integral becomes
\begin{equation*}
\frac{1}{2} \int_{0}^{N} 1 {-} \cos \left( \frac{2(n+1)\pi x}{N} \right) dx.
\end{equation*} 
Taking into consideration that $\sin(\xi \pi) {=} 0, \xi {\in} \mathbb{Z}$ and computing the above integral we get the result
\begin{equation*}
\int_{0}^{N} \sin^2\left( \frac{(n+1)\pi x}{N} \right) dx {=} \frac{N}{2}.
\end{equation*}
Thus we also have
\begin{equation*}
\int_{0}^{N} \sin^2\left( \frac{(m+1)\pi y}{N} \right) dy {=} \frac{N}{2}.
\end{equation*}
So we finally have that the Fourier coefficient for $\overline{F}_{i,nm}(x, y, t)$ is 
\begin{equation}
h_{i,nm}(t) {=} \frac{4}{N^2} \int_{0}^{N} \int_{0}^{N} \overline{F}_i(x, y, t) \phi_{nm}(x, y) dy dx.
\label{eq:fourier_coeff_F}
\end{equation} 
Under the new definition of $\overline{F}_{i, nm}(x, y, t)$ \eqref{eq:time_source_relation} becomes
\begin{equation*}
\sum_{n,m{=}0}^{\infty} \left( T'_{i,nm}(t) {+} \lambda_{i,nm} D_i T_{i,nm}(t) {-} h_{i,nm}(t) \right) \phi_{i,nm}(x, y) {=} 0.
\end{equation*}
The above is an orthogonal relationship which implies that since $\phi_{i,nm}(x, y) {\ne} 0$ and we want to solve for $T_{i,nm}(t)$ that it must be that
\begin{equation*}
T'_{i,nm}(t) {+} \lambda_{i,nm} D_i T_{i,nm}(t) {=} h_{i,nm}(t).
\end{equation*}
This equation is a first order linear ODE which can be solved by multiplying it by an integrating factor, so that the LHS is the result of an application of the product rule and thus can be written as a 
derivate of the product between the dependent term, $T_{i,nm}(t)$, and the integrating factor, after which it can be integrated with respect to (\wrt) the independent variable, $t$. 
Using the integrating factor $e^{\lambda_{i, nm} D_i t}$ and integrating \wrt $t$ we have  
\begin{equation*}
e^{\lambda_{i,nm} D_i t} T_{i,nm}(t) {=} \int_{0}^{t} e^{\lambda_{i,nm} D_i \tau} h_{i,nm}(\tau) d\tau {+} C_{i,nm}, C_{i,nm} {\in} \mathbb{R}.
\end{equation*}
Multiplying the above by $e^{\minus\lambda_{i,nm} D_i t}$ we get the general solution
\begin{equation}
T_{i,nm}(t) {=} e^{\minus\lambda_{i,nm} D_i t} \left( \int_{0}^{t} e^{\lambda_{i,nm} D_i \tau} h_{i,nm}(\tau) d\tau {+} C_{i,nm} \right), C_{i,nm} {\in} \mathbb{R}
\label{eq:time_depen_sol}
\end{equation}
Applying the initial condition (IC) \eqref{eq:ic_inhomo_v}, so to find $C_{i,nm}$ and thus get the particular solution, results in
\begin{equation*}
v_{i, nm}(x, y, 0) {=} \sum_{n,m{=}0}^{\infty} T_{i,nm}(0)\phi_{i,nm}(x, y) {=} g_i(x, y).
\end{equation*}
By the rules of exponents $e^0 {=} 1$ and $\int_{0}^{0} \gamma(\xi) d\xi {=} 0$ it is easily shown that \newline
$T_{i,nm}(0) {=} C_{i,nm}$ and therefore
\begin{equation*}
g_{i,nm}(x, y) {=} \sum_{n,m{=}0}^{\infty} C_{i,nm} \phi_{i,nm}(x, y).
\end{equation*}
Thus $C_{i,nm}$ is the Fourier coefficient of the generalized Fourier series of $g_i(x, y)$ and thus
\begin{equation*}
C_{i,nm} {=} \frac{\int_{0}^{N} \int_{0}^{N} g_i(x, y) \phi_{i,nm}(x, y) dy dx}{\int_{0}^{N} \int_{0}^{N} \phi_{i,nm}^2(x, y) dy dx}.
\end{equation*}
Similarly to $h_{i,nm}(t)$ the denominator is $\frac{N^2}{4}$ and we are left with
\begin{equation}
C_{i,nm} {=} \frac{4}{N^2} \int_{0}^{N} \int_{0}^{N} g_i(x, y) \phi_{i,nm}(x, y) dy dx.
\label{eq:fourier_coeff_g}
\end{equation}
Now all components for the solution of the IBVP \eqref{eq:carcin_pde_inhomo}-\eqref{eq:ic_inhomo} are known and we have
\begin{align}
c_i(x, y, t) &{=} v_{i,nm}(x, y, t) {+} r_i(x, y, t),
\label{eq:carcin_sol}\\
v_{i,nm}(x, y, t) &{=} \sum_{n,m{=}0}^{\infty} T_{i,nm}(t) \phi_{i,nm}(x, y), \tag{\ref{eq:variable_sol}}\\
\phi_{i,nm}(x, y) &{=} \sin\left( \frac{(n+1)\pi x}{N} \right) \sin\left( \frac{(m+1)\pi y}{N} \right), \tag{\ref{eq:spatial_sol}}\\
T_{i,nm}(t) &{=} e^{\minus\lambda_{i,nm} D_i t} \left( \int_{0}^{t} e^{\lambda_{i,nm} D_i \tau} h_{i,nm}(\tau) d\tau {+} C_{i,nm} \right), \tag{\ref{eq:time_depen_sol}}\\
h_{i,nm}(t) &{=} \frac{4}{N^2} \int_{0}^{N} \int_{0}^{N} \overline{F}_i(x, y, t) \phi_{nm}(x, y) dy dx, \tag{\ref{eq:fourier_coeff_F}}\\
\overline{F}_i(x, y, t) &{=} F_i(x, y, t) {-} r_i(x, y, t)_t {+} D_i\Delta r_i(x, y, t), \tag{\ref{eq:source_term_v}} \\
C_{i,nm} &{=} \frac{4}{N^2} \int_{0}^{N} \int_{0}^{N} g_i(x, y) \phi_{i,nm}(x, y) dy dx, \tag{\ref{eq:fourier_coeff_g}}\\
g_i(x, y) &{=} f_i(x, y) {-} r_i(x, y, 0), \tag{\ref{eq:ic_func_inhomo_v}}\\
r_i(x, y, t) {=} &H(y{-}x)A_{i, 1}(y, t) {+} H(x{-}y)A_{i, 2}(x, t) \nonumber\\
&{+} 2H(x{-}N)(B_{i, 1}(y, t) {-} A_{i, 2}(x, t)) {+} 2H(y{-}N)(B_{i, 2}(x, t) {-} A_{i, 1}(y, t)) \nonumber\\
&{+} H(y{-}x{-}N)(A_{i, 1}(y, t) {-} B_{i, 2}(x, t)), \tag{\ref{eq:ref_carcin_concen}}\\
&{+} H(xy{-}N^2)(A_{i, 1}(y, t) {+} A_{i, 2}(x, t) {-} B_{i, 1}(y, t) {-} B_{i, 2}(x, t)) \nonumber\\
&{+} H(x{-}y{-}N)(A_{i, 2}(x, t) {-} B_{i, 1}(y, t)) \nonumber\\
H(\xi) {=} &\begin{cases}
1, &\xi {>} 0 \\
\frac{1}{2}, &\xi {=} 0 \\
0, &\xi {<} 0
\end{cases}, \tag{\ref{eq:heaviside}}\\
\lambda_{i, nm} &{=} \frac{((n+1)^2 {+} (m+1)^2)\pi^2}{N^2}, n,m {=} 0, 1, 2, .... \tag{\ref{eq:sep_const}}
\end{align}

\section{Gene Expression Neural Network}

\section{Cellular Automaton}

\end{document}
