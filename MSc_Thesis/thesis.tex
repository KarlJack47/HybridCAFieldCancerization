% University of Alberta Example Thesis
% By the Rogue's Gallery, Department of Computing Science
% Last updated September 15, 2017

\providecommand{\main}{.}
\documentclass[12pt]{report}          % for default format

%%%%%%%%%%%%%%%%%%%%%%%%%
% Package dependencies  %
%%%%%%%%%%%%%%%%%%%%%%%%%
\usepackage[utf8]{inputenc}
\usepackage{subfiles}
\usepackage[titletoc]{appendix}
\usepackage{amssymb,amsmath,amsthm}
\newtheorem{theorem}{Theorem}[section]
\newtheorem{corollary}{Corollary}[theorem]
\newtheorem{lemma}[theorem]{Lemma}
\usepackage{url}
\usepackage{xr-hyper}
\usepackage[breaklinks=true,hidelinks]{hyperref} % See https://en.wikibooks.org/wiki/LaTeX/Hyperlinks#Customization
\usepackage[table]{xcolor}
\usepackage{tabulary} % Better text wrapping in tables. See https://en.wikibooks.org/wiki/LaTeX/Tables
\usepackage{rotating} % For the 'sidewaystable' environment. See https://en.wikibooks.org/wiki/LaTeX/Rotations
\usepackage{multirow} % For multirow/multicolumn cells in a table. See https://en.wikibooks.org/wiki/LaTeX/Tables#Columns_spanning_multiple_rows
\usepackage{multicol}
\usepackage[font=scriptsize]{caption}
\usepackage[font=scriptsize]{subcaption}
\usepackage{float}
\usepackage{graphicx}
\graphicspath{{\main/images/}}
\usepackage{tikz}
\usetikzlibrary{shapes.geometric, arrows, positioning}
\usepackage{enumitem}   % for more control over enumerate
\usepackage{mdwlist}	% compact lists and the 'note' environment
\usepackage{xspace}
\usepackage{xcolor}
\usepackage{csquotes} % Also used with biblatex
\usepackage{xmpincl} % Seems to be needed when converting to PDF/A
\usepackage{titlesec}
\setcounter{secnumdepth}{4}

\usepackage{\main/uathesis}
\definecolor{hillencolor}{rgb}{0.30,0.03,0.66}
\definecolor{edits}{rgb}{0.13, 0.55, 0.13}
%%%%%%%%%%%%%%%%%%%%%%
% tikzstyle commands %
%%%%%%%%%%%%%%%%%%%%%%
\tikzstyle{startstop} = [rectangle, rounded corners,
                         minimum width=2cm, minimum height=0.5cm,
                         text centered, draw=black, 
                         fill=red!30]
\tikzstyle{io}        = [trapezium, trapezium left angle=70, trapezium right angle=110,
                         minimum width=2cm, minimum height=0.5cm,
                         text centered, draw=black, text width=2cm, 
                         fill=blue!30]
\tikzstyle{process}   = [rectangle, minimum width=0.5cm, minimum height=0.5cm,
                         text centered, draw=black, text width=3.47cm,
                         fill=orange!30]
\tikzstyle{decision}  = [diamond, aspect=6,
                         text centered, draw=black, text width=3.5cm,
                         fill=green!30]
\tikzstyle{arrow}     = [thick,->,>=stealth]

%%%%%%%%%%%%%%%
% biblatex    %
%%%%%%%%%%%%%%%
% (Added by Bernard Llanos)
% biblatex is intended to be the successor to BibTeX
% (https://en.wikibooks.org/wiki/LaTeX/Bibliography_Management#biblatex)
\usepackage[backend=biber,style=nature,backref=true,sortcites=true,sorting=nyt,doi=false,url=true,isbn=false,maxbibnames=3,mincitenames=1,maxcitenames=2]{biblatex}
% `backref=true` adds back references - Links to the in-text citations from
% the corresponding bibliography entries. Back references are not mentioned
% in thesis guidelines, but are, in my opinion, helpful for reading and editing.
\renewcommand*{\bibfont}{\normalfont\scriptsize}
\usepackage[american]{babel}

\defbibenvironment{bibliography}
  {\trivlist}
  {\endtrivlist}
  {\item
   \printtext[labelnumberwidth]{%
   \printfield{labelprefix}%
   \printfield{labelnumber}}%
   \addspace}

% The following macro will put back references on the right edge of the page
% (https://tex.stackexchange.com/questions/149009/biblatex-pagebackref-reference-in-the-flush-right-margin)
%\renewbibmacro*{pageref}{%
%   \iflistundef{pageref}
%     {\renewcommand{\finentrypunct}{\addperiod}}
%     {\renewcommand{\finentrypunct}{\addspace}%
%      \printtext{\addperiod\hfill\rlap{\hskip15pt\colorbox{blue!5}{\scriptsize\pr%                 intlist[pageref][-\value{listtotal}]{pageref}}}}}}

\addbibresource{\main/refs.bib}

%%%%%%%%%%%%%%%%%%%%%%%%%%%%
% Shorthands               %
%%%%%%%%%%%%%%%%%%%%%%%%%%%%

% From the CVPR paper template (http://cvpr2017.thecvf.com/submission/main_conference/author_guidelines)
% Add a period to the end of an abbreviation unless there's one
% already, then \xspace.
\makeatletter
\DeclareRobustCommand\onedot{\futurelet\@let@token\@onedot}
\def\@onedot{\ifx\@let@token.\else.\null\fi\xspace}

\def\eg{\emph{e.g}\onedot} \def\Eg{\emph{E.g}\onedot}
\def\ie{\emph{i.e}\onedot} \def\Ie{\emph{I.e}\onedot}
\def\cf{\emph{c.f}\onedot} \def\Cf{\emph{C.f}\onedot}
\def\etc{\emph{etc}\onedot} \def\vs{\emph{vs}\onedot}
\def\wrt{w.r.t\onedot} \def\dof{d.o.f\onedot}
\def\etal{\emph{et al}\onedot}
\def\minus{\text{-}}
\makeatother

%%%%%%%%%%%%%%%%%%%%%%%%%%%%%%%%%%%%%%%%%%%
% Title page and Table of Contents Tweaks %
%%%%%%%%%%%%%%%%%%%%%%%%%%%%%%%%%%%%%%%%%%%

%% Correct title for TOC
\renewcommand{\contentsname}{Table of Contents}
% Fill in the following
\title{Hybrid Cellular Automaton for Field Cancerization}
\author{Karl Deutscher}
\degree{\MSc}
\dept{Mathematical and Statistical Sciences}
\field{Applied Mathematics}
\submissionyear{\number2022}

%%%%%%%%%%%%%%%%%%%%%
% Document Content  %
%%%%%%%%%%%%%%%%%%%%%

% This is a modular document.
% The 'subfiles' package allows you to typeset the included
% documents separately from the main document, so that you
% can view only pieces of the thesis at a time.
% See https://en.wikibooks.org/wiki/LaTeX/Modular_Documents
%
% Subfiles that contain references: You can just run
% `biber subfilename` on them when compiling them individually.
% There is no need to make them reference the bibliography database
% 'refs.bib', as they inherit the reference from this file.

\newcommand{\onlyinsubfile}[1]{#1}
\newcommand{\notinsubfile}[1]{}

\begin{document}

\renewcommand{\onlyinsubfile}[1]{}
\renewcommand{\notinsubfile}[1]{#1}

\preamblepagenumbering % lower case roman numerals for early pages
\titlepage % adds title page. Can be commented out before submission if convenient

\subfile{\main/tex/abstract.tex}

\doublespacing
% \truedoublespacing
% \singlespacing
% \onehalfspacing

\subfile{\main/tex/dedication.tex}
\subfile{\main/tex/acknowledgements.tex}

\singlespacing % Flip to single spacing for table of contents settings
               
\tableofcontents

\listoffigures
\listoftables

% The rest of the document has to be at least one-half-spaced.
% Double-spacing is most common, but uncomment whichever you want, or 
% single-spacing if you just want to do that for your personal purposes.
% Long-quoted passages and footnotes can be in single spacing
\doublespacing
% \truedoublespacing
% \singlespacing
% \onehalfspacing

\setforbodyoftext % settings for the body including roman numeral numbering starting at 1

\subfile{\main/tex/Chapter1Introduction/biological/biological_overview.tex}
\subfile{\main/tex/Chapter1Introduction/biological/carcinogenesis.tex}
\subfile{\main/tex/Chapter1Introduction/biological/csc.tex}
\subfile{\main/tex/Chapter1Introduction/biological/field_cancerization.tex}
\subfile{\main/tex/Chapter1Introduction/mathematical/math.tex}
\subfile{\main/tex/Chapter2Model/overview.tex}
\subfile{\main/tex/Chapter2Model/carcinogen_pde.tex}
\subfile{\main/tex/Chapter2Model/gene_expression_nn.tex}
\subfile{\main/tex/Chapter2Model/ca.tex}
\subfile{\main/tex/Chapter3ApplicationAndImplementation/overview.tex}
\subfile{\main/tex/Chapter3ApplicationAndImplementation/hnscc.tex}
\subfile{\main/tex/Chapter3ApplicationAndImplementation/application.tex}
\subfile{\main/tex/Chapter3ApplicationAndImplementation/implementation.tex}
\subfile{\main/tex/Chapter4Results/overview.tex}
\subfile{\main/tex/Chapter4Results/equilibrium.tex}
\subfile{\main/tex/Chapter4Results/general_observations.tex}
\subfile{\main/tex/Chapter4Results/grid_size.tex}
\subfile{\main/tex/Chapter4Results/carcinogens.tex}
\subfile{\main/tex/Chapter4Results/tumour_excision.tex}
\subfile{\main/tex/Chapter4Results/cell_lineages.tex}
\subfile{\main/tex/conclusion.tex}

% Renaming the bibliography: See http://tex.stackexchange.com/questions/12597/renaming-the-bibliography-page-using-bibtex
\renewcommand\bibname{References}
\clearpage\addcontentsline{toc}{chapter}{\bibname}

\singlespacing
%\setlength{\columnsep}{0.2cm}
%\begin{multicols*}{2}[\LARGE\center{\textbf{References}}]
%\printbibliography[heading=none]
%\end{multicols*}
%\nocite{*}
\printbibliography

\normalsize
\doublespacing

\end{document}
