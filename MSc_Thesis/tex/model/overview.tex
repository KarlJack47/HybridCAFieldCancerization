\providecommand{\main}{../..}
\documentclass[\main/thesis.tex]{subfiles}

\begin{document}

\chapter{Model}

\section{Model Overview}

The hybrid cellular automaton (CA) model proposed is one that simulates field 
cancerization. It is a hybrid model as the CA rule depends on the output of 
other mathematical objects. In this case the two mathematical objects are 
partial differential equations (PDE) and neural networks (NN). The PDE models 
spread of one or more carcinogen(s) within the domain of the CA. The NN is used 
to compute the change in gene expression of the genes under consideration for 
each cell with respect to the amount of carcinogen at the cell's location and 
it's age. The CA includes states that are used to represent the following 
biological cell types: normal tissue cells (NTC), mutated normal tissue cells 
(MNTC), normal stem cells (NSC), mutated normal stem cells (MNSC), cancer stem 
cells (CSC), and tumour cells (TC). Evolution of the model occurs in the 
following basic steps:
\begin{enumerate}
	\item Carcinogens are allowed to spread via the PDE.
	\item Changes in gene expressions resulting from carcinogenesis exposure and 
	      age of the cell are computed by the NN and gene mutations are allowed 
	      to occur. 
	\item All the cells state and movement is updated using the CA rule. 
\end{enumerate}

\end{document}
