\providecommand{\main}{../../..}
\documentclass[\main/thesis.tex]{subfiles}

\begin{document}

\subsection{Cancer Stem Cells}

Before discussing cancer stem cells (CSCs) it should be noted that there is no standardized definition of CSCs but instead many slightly different ones, some of which even contradict each other. Typically the definition of CSCs composed by researchers is such that it suits their current work. CSCs are not SCs, they are cells that have some of the characteristics of SCs.

SCs are biological cells that can differentiate into other types of cells and divide to reproduce more of the same type of SCs. There are two types of SCs: embryonic and somatic (adult). In the case of carcinogenesis we only consider somatic SCs, for simplicity we will term a somatic SC as a normal SC (NSC). These are found in all tissues, particularly bone marrow, fat cells, and blood, in which their function is to maintain and repair the tissue. NSCs undergo two types of cell division namely, symmetric (produces two identical SCs) and asymmetric (produces one SC and a progenitor cell) \parencite{Beckmann}. A cell is a SC if it has the following two properties: self-renewal and potency (potential to differentiate into different cell types).

In the context of SCs, self-renewal is considered the ability to achieve numerous cycles of cell division while maintaining the undifferentiated state. One way that self-renewal can be obtained is through obligatory asymmetric replication. Another mechanism of self-renewal is stochastic differentiation, which occurs when simultaneously one SC develops into two differentiated daughter cells while another SC undergoes symmetric division. Progenitors move through several rounds of cell division before terminally differentiating into a mature cell \parencite{Beckmann}. 

Types of potency include totipotent (omnipotent), pluripotent, multipotent, oligopotent, and unipotent. Totipotent SCs can differentiate into embryonic and extra-embryonic cell types, which results in the construction of a complete organism \parencite{Scholer}. Pluripotent SCs are descendants of totipotent cells and can differentiate into nearly all cells, \ie, cells derived from any of the three germ layers (endoderm, mesoderm, ectoderm) \parencite{Scholer}. Pluripotent NSCs are rare and small in number but they can be found in tissues. Multipotent SCs can differentiate into a number of cell types, but only those of a closely related family of cells \parencite{Scholer}. Unipotent cells can produce only one cell type, their own, but have the property of self-renewal \parencite{Scholer}. Most NSCs are multipotent and named based upon their tissue of origin. 

Cancer stem cells (CSCs) are multipotent cells in a tumour that like NSCs have self-renewal, but in addition, have the abilities of tumour initiation, migration and metastasis \parencite{Biddle,Bu}. Another definition, is a small population of cells within the tumour that are tissue specific, slow dividing and with unlimited self-renewal capacity \parencite{Cabanillas}. CSCs are critical in tumour initiation and progression, through their interaction with cancer cells and the extracellular matrix \parencite{Catalano}. A CSC differs from an NSC in that it has deregulated proliferative capacity and can have metastatic properties \parencite{Cabanillas,Zhou}. 

The origin of CSCs is explained by three possible processes. The first process states that an NSC undergoes several genetic as well as epigenetic alterations to give rise to a CSC \parencite{Feller}. The second process states that CSCs originate from NSCs that acquire a precancerous phenotype during its' development stage \parencite{Bjerkvig,Feller,GonzalezMoles}. The third process states that the CSC originate from mature tumour cells \parencite{DiFiore,HerrerosVillanueva,Kumar,Moon} or epithelial cells \parencite{Bjerkvig,Feller,GonzalezMoles} that undergo dedifferentiation into a SC through modifications in signaling pathways and regulatory mechanisms. Note that the first and second processes only differ in whether an NSC acquires a genetic alteration when it is fully developed or still in development.

\end{document}