\providecommand{\main}{../../..}
\documentclass[\main/thesis.tex]{subfiles}

\begin{document}

\subsection{Carcinogenesis}

Most carcinogenesis models consider that cancer is initialized from the result 
of a multistep process. The thought is that a normal cell doesn't become a 
cancer cell until multiple genetic alterations accumlate within it. The number 
of genetic alterations in a cancer cell is an indictator to which level of 
malignancy the cell is at. 

\textcite{Gatenby} found six microenvironmental barriers for a 
malignant phenotype: apoptosis with loss of basement membrane contact, 
inadequate growth promotion, senescence (deterioration of a cells' power of 
division and growth with age), hypoxia (deficiency in the amount of oxygen 
reaching the tissues), acidosis (excessively acidic condition of the body fluids 
or tissues), and ischaemia (restriction of blood supply to tissues, causing 
hypoxia). The development of cancer occurs when a normal cell overcomes at least 
one of these barriers. Thus, the microenvironment is an important factor to 
consider in cancer initialization. 

A normal cell lineage can acquire epimutations, termed mutations in 
\textcite{Curtius}, that are positively selected in the microenvironment of a 
healthy organ. Not only do carcinogens cause mutations, the natural aging 
process of tissue can as well because mutations accrue in such tissues 
\parencite{Blokzijl}. If you consider that all mutant cells are cancerized then 
the entire body will become increasingly cancerized as it grows older 
\parencite{Curtius}. A driver mutation is one that confers growth or survival 
advantages for tumour cells within the appropriate microenvironment 
\parencite{Calabrese,MGreaves,Stratton}. A passenger (neutral) mutation is one 
that passively accumulates in cell lineages 
\parencite{Calabrese,MGreaves,Stratton}. It may be that some driver mutations 
are not currently affecting cancer growth but instead had previously driven the 
growth of an ancestral lineage \parencite{Curtius}. Progression to cancer 
usually requires the accumulation of multiple driver mutations 
\parencite{Weaver}. A mutant lineage/clone, can grow to produce large patches, 
or fields, of cells that are predisposed to eventually progress to neoplasm. 

The stromal microenvironment is a key regulator of self-renewal in the 
epithelium \parencite{Davis}. Epithelium is one of the four basic types of 
animal tissue, along with connective, muscle and nervous. Epithelial tissues 
line the outer surfaces of organs and blood vessels throughout the body, as well 
as the inner surfaces of cavities in many internal organs. Epithelial tissue is 
organized into two different types: glandular and squamous. A gland is one or 
more cells that produce and secrete a specific product. Glandular tissue has two 
types: exocrine (secrete their products (enzymes, muscus, milk, \etc) into ducts 
that lead directly to the external environment), endocrine glands (secrete their 
products (hormones) directly into the bloodstream). Exocrine glandular 
epithelium tissues include salivary glands, the esophagus, gastric glands, 
intestinal glands, pancreatic glands, mammary glands, and sweat glands. Examples 
of endocrine glandular tissues include pituitary glands, thyroid glands, 
parathyroid glands, adrenal glands, pancreas, gonads, and pineal glands. 
Squamous epithelium tissue is a single layer of flat cells in contact with the 
basal lamina of the epithelium. It is often permeable and occurs where small 
molecuels need to pass quickly through membranes via filtration or diffusion. 
Some examples of squamous tissue include the skin, walls of capillaries, linings 
of the pericardial, pleural cavities, peritoneal cavities, and linings of the 
alveoli of the lungs. The tissue architecture constrains evolution by limiting 
the ability of mutant clones to expand \parencite{Martens}, so understanding the 
differences between mutant clone expansions in glandular and squamous tissue is 
important. In glandular epithelium a mutant clone undergoes niche successions, 
where a mutant stem cell (SC) in the gland replaces all other SCs; after which 
the mutant glands produces a field of mutant glands by gland fission 
\parencite{Baker,LGreaves,McDonald,Nicholson}. Epithelial cells within a 
squamous tissue expand by basal replacement of neighbouring SCs 
\parencite{Klein,Alcolea,Clayton,Doupe}. During normal homeostasis basal cells 
proliferate to prodcue differentiated progeny that then form the superficial 
layers of the epithelium \parencite{Teixeira}. Any new basal cells compete to 
grow a patch by lateral replacement of one progenitor cell by another 
\parencite{Teixeira}. 

Phenotypic change is caused by either individual high-impact mutations or 
epistasis among a group of mutations that require each other 
\parencite{Curtius}. As well, large-scale mutational events that simultaneously 
alter expansive parts of the genome will elicit phenotypic change 
\parencite{Li,Stachler}. Microenvironmental factors provide selective pressures 
for phenotype adaptation in which cells explore the adaptive landscape (via 
genetic mutations or phenotypic plasticity), and genotypes of lineages reflect 
the phenotype that survived \parencite{Curtius}. \textcite{Curtius} consider 
only phenotypes that are cancer-related when discussing cancer development and 
initiation, thus they exclude regions of tissue having DNA damage or passenger 
mutations and oncogene or tumour supressor gene mutations that are not currently 
active. It is important to note that cancerized phenotypes may be subtle and/or 
lasting only for a short time \parencite{Curtius}, making them hard to detect. 
Phenotypic consequences of a driver mutation may be context dependent, so the 
existence of a driver mutation may not be sufficient to cause a phenotypic 
change in the current microenvironmental condition \parencite{Curtius}. Due to 
the fact that crypt fission is the main mechanism in glandular tissue 
development, a mutation that increases upregulation of crypt fission is a cause 
for carcinoma development or initialzation \parencite{Curtius}. Mutations that 
help cells adhere to the basal membrane influence carcinoma development in 
squamous tissue because they will not allow those cells to migrate 
and differentiate, thus causing the mutated cells to expand and/or replace 
neighbouring cells \parencite{Curtius}.

\end{document}