\providecommand{\main}{../../..}
\documentclass[\main/thesis.tex]{subfiles}

\begin{document}
 
\subsection{Field Cancerization}
 
The idea of field cancerization was first mentioned by \textcite{Slaughter}
in 1953 when histologically observing 783 squamous-cell tumours in oral 
cancers. Within the entire patient population it was found that benign 
epithelium surrounding the malignant tumour was abnormal. As well some of the 
patients had multiple separate tumours occur in the same area of the oral 
cavity. From these observations they proposed a process termed field 
cancerization, in which a carcinogenic agent preconditions an area of 
epithelium towards cancer. If a carcinogenic agent is exposed to an area of 
epithelium for a sufficient amount of time and with enough intensity then it 
produces irreversible changes in cells and cell groups such that the process 
toward cancer becomes inevitable \parencite{Slaughter}. \textcite{Slaughter} 
also hypothesized that a field of preconditioned epithelium may develop cancer 
at multiple points and possibly lead to multiple tumours. As a result they 
don't believe cancer arises from one cell that suddenly becomes malignant but 
instead from areas of precancerous change. From their hypotheses 
\textcite{Slaughter} consider it may be that local recurrence after surgery or 
radiation occurs due to left-over benign epithelium that is preconditioned 
towards cancer, \ie, from the remaining field. Many papers were written 
following \textcite{Slaughter} that showed field cancerizaiton can be found in 
colon carcinoma in patients with 
\parencite{Galandiuk,Leedham,Koizumi,VanDekken2006} IBD (irritable bowl 
syndrome) and without IBD 
\parencite{Alonso,Asada,Damania,Hawthorn,Kamiyama,Kaz,Milicic,Shen}, gastric 
carcinoma 
\parencite{GutierrezGonzalez,Kang,McDonald,Takeshima,Ushijima,Yamanoi,Zaky}, 
aesophogeal squamous carcinoma 
\parencite{Cense,Kammori,Lee,Matsuda,Oka,RoeschEly,Yakoub}, aesophgeal 
adenocarcinoma \parencite{Galipeau,Maley052004,Maley102004,VanDekken1999}, 
non-small-cell lung squamous carcinoma 
\parencite{Chang,Franklin,Kadara2012,McCaughan,Pipinikas,Sozzi,Steiling},
non-small-cell lung adenocarcinoma 
\parencite{Gomperts,Kadara2014,Lin,Nakachi,Weichert},
small-cell lung carcinoma 
\parencite{Gomperts,Kadara2014,Lin,Nakachi,Wirtschafter},
head and neck squamous cell carcinoma (HNSCC) (oral, oropharanx, hypopharanx, 
larynx) 
\parencite{Slaughter,Braakhuis,Angadi,BoscoloRizzo,Califano1996,Narayana,Pentenero,Shaw,VanDerVorst}
, breast carcinoma 
\parencite{Dworkin,Ellsworth,Foschini,Rivenbark,Trujillo2011}, cervix 
\parencite{Chu}, prostate carcinoma \parencite{Haaland,Nonn,Trujillo2012}, 
bladder carcinoma \parencite{Hafner,Vriesema,Wang}, skin carcinoma 
\parencite{Hu,Kanjilal,Stern,Szeimies,Vatve}, melanoma \parencite{Shain}, and 
blood cancer \parencite{Mori,MGreaves,Genovese}.
 
At the time \textcite{Slaughter} were conducting their research, the study of 
genetics was in its infancy, so they could only create hypotheses based on 
histological observations. The desire to understand cancer from a molecular 
perspective brought about studies using different molecular analyses on 
tumour-adjacent tissue to discover biomarkers that would indicate the presence 
of a field. Biomarkers that were discovered to correlate with the presence of a 
field are loss of heterozygosity (LOH) \parencite{Tabor062001}, microsatellite 
alterations \parencite{Tabor062001}, chromosomal instability 
\parencite{Hittelman}, and mutations in the TP53 gene 
\parencite{Brennan,VanHouten}. \textcite{Braakhuis} attempted to explain field 
cancerization from the perspective of genetics. They altered 
\textcite{Slaughter} definition of field cancerization to: growth of one or more 
genetically altered cell(s) that produces a field of cells predisposed to 
subsequent tumour growth. \textcite{Braakhuis} enforce the understanding that a 
field lesion does not grow invasively nor does it have metastatic properties. 
They proposed a process for the formation of a field and subsequent tumour 
within it for head and neck mucosa, esophagus, and bladder carcinomas. First, a 
SC acquires one or more genetic alterations and forms a group of cells with a 
mutation in TP53 (clonal unit) that creates genetically altered daughter cells. 
The SC and its' clonal unit is considered to be a lesion. Following more genetic 
alterations the SC gains growth advantage and develops into an expanding clone. 
The lesion, which is gradually becoming a field, displaces the normal epithelium 
surrounding it due to the enhanced proliferative capacity of a genetically 
altered clonal unit. As the lesion becomes larger, additional genetic hits 
create various sub-clones (clonal divergence) within the field. Eventually a 
sub-clone evolves into invasive cancer due to the presence of a large number of 
genetically altered SCs, and clonal divergence/selection. 

Based on genetic evidence there currently exists two main hypotheses that 
explain the underlying cellular basis of field cancerization: polyclonal origin 
and monoclonal origin. Polyclonal origin proposes that mutations occur in 
multiple sites of the epithelium due to continuous carcinogen exposure which 
leads to multi-focal carcinomas or lesions of independent origin 
\parencite{VanOijen}. Monoclonal origin proposes that the mutant cells from the 
initial lesion migrate and develop multiple lesions that share a common clonal 
origin. Three theories have been proposed to explain the mechanisms involved in 
monoclonal origin. The first two theories suggest that some tumour or tumour 
progenitor cells from the primary site either migrate through the submucosa or 
shed in the lumen of an organ (\eg, the oral cavity or the bladder) which in 
both cases leads to the formation of a tumour at an adjacent secondary site 
\parencite{Califano1999,Bedi}. The third theory suggests that the continuous 
genetically altered lesions in the epithelium lead to the development of 
clonally related neoplastic lesions that develop via lateral spreading in the 
same or adjacent anatomical areas \parencite{Angadi,Prevo,Tabor2002,Simon}. Some 
biological researchers believe both hypotheses hold, while others that just one 
holds. \textcite{Braakhuis} are in the group that believes only monoclonal 
origin is the correct hypothesis. Histologically it has been suggested that
monoclonal origin holds true due to multiple biopsies sharing \textquote{early 
markers of carcinogenesis} \parencite{VanHouten,Califano1999,Tabor122001}. 
Also the \textquote{"late"} markers being heterogeneously mixed within the 
tissue in and surrounding the tumour implies that clonal divergence, \ie, 
development of multiple subclones occurred \parencite{Nowell}, which means that 
monoclonal origin likely occurred. Though monoclonal and polyclonal 
origin are the standard hypotheses to explain field cancerization origin based 
upon genetic evidence, these do not fully address all mechanisms for field 
formation. In fact, further exploration of biological mechanisms would likely 
elicit expanded theories. 

Another breakthrough in biology since \textcite{Slaughter} was the discovery 
of CSCs and their importance in cancer initiation, progression, and treatment. 
\textcite{Simple} came up with a model to explain field cancerization using 
\textcite{Braakhuis} model plus the addition of CSCs. They define field 
cancerization as the occurrence of molecular abnormalities in the tumour 
adjacent mucosal field. They consider both monoclonal and polyclonal origin 
within their model. \textcite{Simple} model for oral cancer includes the 
following steps. First, continuous exposure of the oral mucosa to carcinogens 
results in molecular alterations that lead to the induction of CSC-like 
behaviour in a stepwise manner. Second, CSCs originate either by transformation 
of the NSCs or be de-differentiation of the tumour cells and migrate through 
normal mucosa to develop the field. Third, initial hits at 17p (TP53) and 3p/9p 
(p16/FHIT) lead to transformation of the NSCs into transient amplifying cells 
(TACs). Fourth, these transformed cells divide and expand to create a field of 
neoplastic cells. Fifth, a genetic hit in the cells within the field at 13q, 
location of the Rb gene, allows a carcinoma to develop. Note that alteration to 
the Rb gene is known to release CSCs from their quiescent stage such that 
proliferation, self-renewal and formation of tumours can occur. 

The development of the field mentioned at the second and fourth steps of the 
process either occurs polyclonally or monoclonally. In the case of polyclonal 
origin the following process occurs. First, NSCs at different sites in the 
mucosa undergo stepwise transformation into CSCs through independent 
carcinogen-mediated molecular alterations. Second, the CSCs proliferate leading 
to the development of clones at different sites. Third, additional genetic hits 
give rise to further divergence in the sub-clones within the field. Fourth, one 
of the sub-clones obtains the final genetic hit at 13q to develop into 
carcinoma. Considering the monoclonal process of field cancerization the 
following occurs. An initial lesion originates from the NSCs and gradually 
expands to become a field. Next, either the resident CSCs of the lesion or those 
that originate by de-differentiation of the tumour cell migrate from the 
parental lesion. The de-differentiation process can be driven by mutations in 
TP53 and over expression of OCT4, SOX2. The CSCs that migrate have gained growth 
advantage and thus can displace the normal epithelium by either lateral 
intra-epithelial migration or submucosal spread. It is known that the migratory 
CSCs can then switch to the non-migratory form and generate secondary tumours in 
the adjacent mucosal field. 

Recently \textcite{Curtius} decided to study field cancerization from an 
evolutionary perspective. They define a cancerized field to be a single cell or 
group of cells that are further along an evolutionary path towards cancer. Since 
a cancerized field has mutational diversity it is a great candidate for natural 
selection, meaning that over time the fittest mutant clone will dominate the 
field. A cancerized field can be described by the following phenotypic 
properties: growth and death rate, and immune evasion capacity. 
\textcite{Curtius} define field cancerization drivers as mutations that drive 
phenotypic changes that cause a cancerized field. Driver mutations have been 
found in both the carcinoma and the cancerized field thus indicating that a 
driver mutation may also be a field cancerization driver. 

\textcite{Curtius} define field cancerization as a somatic evolutionary 
process that produces cells that are close to cancer. \textcite{Braakhuis} 
definition of field cancerization implies that the mutant clone that grows has 
an altered phenotype that drives its expansion, from an evolutionary perspective 
this is achieved by a mutant clone being fitter than the resident cell 
population. During field cancerization multiple phenotype states are achieved. 
Per \textcite{Curtius} a cancerized fields' formation is driven by exposure 
of a prevalent carcinogen and promoter of clonal expansion and/or subsequent 
convergent evolution of the epigenome. As a result field cancerization can occur 
because of multiple independent clonal expansions, \ie, polyclonal origin. Thus, 
both \textcite{Simple} and \textcite{Curtius} consider that a cancerized field 
can be formed via monoclonal or polyclonal origin.

The general framework that \textcite{Curtius} propose for the initiation of a 
cancerized field is as follows. A group of cells, gradually becoming a 
cancerized field, undergo mutations that occur due to DNA replication errors 
during aging and/or carcinogens, resulting in many genetically distinct clones 
within. As more time passes daughter clones with phenotypes that increase their 
fitness will dominate the group. Finally, if the mutagenic insult is ongoing 
then new clones will be continually generated and the cancerized field will 
appear genetically diverse. At this point carcinoma will first occur as 
described by \textcite{Simple}, \ie, when one of the clones acquires a 
genetic hit at 13q.  

\end{document}