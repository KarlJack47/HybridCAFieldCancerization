\providecommand{\main}{../../..}
\documentclass[\main/thesis.tex]{subfiles}

\begin{document}

\section{Mathematical Literature Review}

% Add references for each
There exists lots of literature that studies cancer initiation, progression, 
metastasis, treatment (chemotherapy, immunotherapy, radiation), 
and effects of various microenvironmental and external factors on cancer 
development from the perspective of mathematical analysis. However, the only 
mathematical analysis on field cancerization that could be found at the time of 
writing were spatial stochastic models in \textcite{Foo} and \textcite{Ryser}.

% Foo review
\textcite{Foo} describe field cancerization as the the process of primary 
tumours forming from genetically altered fields of premalignant cells that have 
high chance of progression to malignancy. Also they state that the premalignant 
fields can cause recurrent tumours if not excised with the primary tumour during 
surgery. The main objectives of their study were to:
\begin{enumerate}
  \item develop a spatial evolutionary framework for field cancerization;
  \item describe the size and geometry of the premalignant fields at the moment 
        of tumour initiation;
  \item determine the risk of multifocal lesions, recurrence timing, and clonal 
        origin of recurrent tumours;
  \item discover the effects of different characteristics of tissue and cancer 
        type on 2 and 3.
\end{enumerate}
The domain of their model is a lattice of dimension ($d$) 1, 2, or 3 wherein 
each lattice point is occupied by one cell. The dimension is generally either 1 
or 2 because although epithelial tissues are 3D it is sufficient to consider 
just 1 or 2D approximations of the areas of interest. For example due to the 
ratio of tube radius to length mammary ducts of the breast, renal tubules of the 
kidney and bronchi tubes of the lung can be approximated with $d{=}1$. Also 
cancer initiation in the squamous epithelium of the cervix, bladder, and oral 
cavity can be considered a $d {=} 2$ process as it occurs in the basal layer 
which is only 1-2 cells thick. Each cell reproduces by placing its 
offspring randomly in one of the 2d neighbours at the rate equal to their 
fitness, $s {\ge} 0$, at exponential waiting time. A cell having a fitness 
advantage has increased reproductive rate or avoidance of apoptotic 
signals. The cell type is determined by the fitness, $s$, which they also 
describe as the number of genetic hits a cell has accumulated. A cell of type-0 
have fitness normalized to 1, $s {=} 1$, and are considered to be wild-type or 
normal. The type-0 cell can become a type-1 cell by acquiring the first type of 
mutation at the rate $u_1$. In general a type-$i$ cell has a fitness advantage 
of $1 + s_i$ relative to the type-(i-1) and is assumed to have acquired all the 
mutations up to mutation $i$. If it acquires the $(i+1)-th$ mutation at rate 
$u_{i{+}1}$ then it will become a type-(i+1) cell. The mutation rates are quite 
small, $u_i << 1$ and so the domain has to be large so that their higher chance 
a cell will acquire a mutation. The lattice is initialized with cells having 
type-0. It is assumed that when a cell has developed $k$ mutations and so is of 
type-$k$ that cancer has initialized and the simulation is stopped. The number of 
mutations $k$ is set based upon the type of cancer because each cancer arises 
from a mutational pathway to malignancy. It is important to note that the model 
assumes all $k$-mutations positively influence cancer initialization and have 
the effect of increased cell growth and/or reduction in apoptotic signalling. 
Another assumption to notice is that all their mutations come from random 
mutations which occur very rarely and are generally fixed by the bodies DNA 
repair process before they have an effect. \textcite{Foo} ignore the genetic 
hits that cause selective disadvantage because it causes the cell and its' 
progeny to die fast. Cell death and reproduction dynamics are accomplished 
similiarly to the biased voter process. The starting time is at the end of 
tissue development and start of the tissue renewal phase. Since this time is 
difficult to acquire it has challenging to find the time of cancer initiation, 
$\sigma_k$.

To determine the probability of a mutant clone population of type-1 
surviving as well as some size properties they considered a one mutation 
model with cells of type-0 having fitness 1 and type-1 having fitness 1+s. The 
lattice is initialized with all type-0 cells except the cell at $x{=}0$ so that 
if the lattice isn't infinite then one obtains a Williams-Bjerknes Model and 
otherwise a biased voter model. The number of type-1 cells as a jump process is 
a discrete time random walk that moves one up with probability $\frac{s}{1{+}s}$ 
or one down with probability $\frac{1}{1{+}s}$ due to the fact that the only 
possible events are type-0 get replaced by type-1 (jump up) and type-1 gets 
replaced by a type-0 (jump down). Thus the probability that a mutant clone 
population of type-1 survives is $\frac{s}{1{+}s} \approx s$ (when $s << 1$). A 
mutant clone population with fitness $s$ is successful if it reaches size
$>> \frac{1}{s}$ as this results in a negligible chance of its extinction. 
Unsuccessful type-1 mutant clone populations typically have a space-time volume 
of order:
$l(s) {=} \begin{cases}
            s^{-2}, & d{=}1 \\
            s^{-1}\ln(s^{-1}), & d{=}2 \\
            s^{-1}, & d{=} 3 
          \end{cases}$.
The probability of a mutant clone population of type-1 surviving still holds for 
spatial Moran models as long as $\frac{1}{u_1} >> l(s)^{\frac{d{+}2}{2}}$. 
Also as a result of the previous condition if the number of type-1 cells is 
significantly less than the the total number of cells, $N$, for all of time then 
successful type-1 mutations arrive as a Poisson process at rate 
$\frac{Nu_1s}{1{+}s}$. Upon considering a mesoscopic model they let type-1 
mutations arrive as Poisson with rate $Nu_1$, distributed uniformly at random in 
the domain. Each mutation event has two potential outcomes:
\begin{itemize}
  \item[(a)] With probability $\frac{s}{1{+}s}$ the mutation is successful and 
             the clonal expansion is approximated with a ball whose radius grows 
             deterministically with macroscopic growth rate $c_d(s)$.
  \item[(b)] With probability $\frac{1}{1{+}s}$ the mutation is unsuccessful and 
             the clone evolves via full stochastic model dynamics conditioned 
             on extinction.
\end{itemize}

Having acquired the basic probabilistic and size aspects of the model they moved 
to a two mutation model so that it could be fit in the context of field 
cancerization whereby type-0 cells have fitness 1, type-1 cells are premalignant 
with fitness $1 {+} s_1$ relative to type-0, and type-2 are cancer cells with 
fitness $1 {+} s_2$ relative to type-1. If $s_1{=}s_2{=}s > 0$ the timing of 
cancer initiation is controlled by the limiting value of the meta-parameter: 
$\Gamma {=} (Nu_1s)^{d{+}1}(c_d^d(s)u_2s)^-1$. Varying the meta-parameter 
$\Gamma$ results in the following three regimes:
\begin{itemize}
  \item[(a)] Regime 1 (R1): When $\Gamma {<} 1$ the first successful type-2 
             mutation occurs within the expanding clone of the first successful 
             type-1 mutation. The time at cancer initiation, $\sigma_2$, is 
             exponential and doesn't depend on the spatial dimension.
  \item[(b)] Regime 2 (R2): For $\Gamma {\in} (10, 100)$ the first successful 
             type-2 mutation occurs within one of the several successful type-1 
             clones. The time at cancer initiation, $\sigma_2$, is not 
             exponential and depends on the spatial dimension.
  \item[(c)] Regime 3 (R3): When $\Gamma {>} 1000$ the first successful type-2 
             mutation occurs after many successful type-1 mutations. It can 
             arise from a successful or unsuccessful type-1 mutation. The time 
             at cancer initiation, $\sigma_2$, is a mixture distribution of 
             these events.
\end{itemize}
The borderline regimes $R1/R2$ and $R2/R3$ occur for $\Gamma {\in} [1, 10]$ 
and $\Gamma {\in} [100, 1000]$, respectively. For all further analysis 
\textcite{Foo} assumed that in $R3$ successful type-2 mutation arise only from 
successful type-1. As the number of cells increases the time that the first 
type-2 mutation arises, $\sigma_2$, decreases and thus cancer initiation occurs 
earlier. When $\sigma_2$ is small there is a diffuse premalignant field and a 
large number of independent lesions. However, when $\sigma_2$ is large there is a 
single premalignant field that contains the initial tumour cell. If one assumes 
that the time of diagnosis, $T_D$, is independent of $\sigma_2$ then the 
premalignant field at time of diagnosis, $\sigma_2 {+} T_D$, can be 
characterized by the field at $\sigma_2$ together with the distribution of 
$T_D$. The size distribution of an initiating clone is the distribution of a 
size-biased pick from the different clones present at the time the initiated 
mutation arises. They found that tumours appearing later have a higher recurrence 
probability if only the malignant portion is removed during surgery. The model 
is moves towards R2 and R3 as $u_1$ increases in which the premalignant field is 
made of an increasing number of independent type-1 patches. As a result of the 
number of type-1 patches increasing so does the chance of a type-2 cell being 
formed and hence $\sigma_2$ decreases. As $u_2$ increases the model moves 
towards R1 in which fewer type-1 clones are required to produce the first 
successful type-2 cell, and the size of the type-1 field decreases. In R1 and R2 
the total distant field size is of the same order of magnitude of the local 
field size, whereas in R3 it is significantly larger. Thus, secondary tumour 
recurrences for cancer types in R3 are more likely to come from the distant 
field and as a result are clonally unrelated to the primary tumour. In R1 the 
expected number of small clones peaks and then declines as larger clones begin 
to dominate, whereas in R2 and R3 small and large clones coexist for a longer 
period of time. A bigger premalignant field increases the chance of fast 
recurrence. In R1 local recurrence is more likely, however when 
$\sigma_2$ is large then it is slightly more likely to come back in the distant 
fields. In R2 and R3 the overall probability of local and distant recurrences is 
comparable, however when $\sigma_2$ is small recurrence is more likely to occur 
in distant distant fields and when $\sigma_2$ is larger local recurrence is more 
likely. \textcite{Foo} found the: distribution of the local field radius, 
distribution of the area of the local field at $\sigma_2$, size-distribution of 
the distant field clones at $\sigma_2$, number of field patches is distributed 
as a mixture of a Poisson and a shifted Poisson random variable, probability of 
a second field tumour having formed before some time, and the probability that 
the distant field at the time of initiation gives rise to a second primary 
tumour at some time. Some of the limitations of \textcite{Foo} are that they: 
use a specific sequence of genetic alterations with specified $u_i$ and $s_i$, 
assume the micro-environment is static and uniform, use circular growth for the 
clones, assume essentially a two-hit mutation model, only consider proliferation 
and apoptosis, ignore tissue structures, and include only normal somatic cells. 

% Ryser review
\textcite{Ryser} applied the model described in \textcite{Foo} to head and neck 
squamous cell carcinoma (HNSCC) which is commonly a result of field 
cancerization. They specifically looked at HPV-negative cases, patients that 
had a history of smoking, and age of diagnosis. \textcite{Ryser} attribute the 
difficulty of detecting the premalignant fields surrounding the primary tumour 
in clinical practise to be a lack of understanding of the dynamics and geometry 
of the fields. Since HNSCC are found in stratified squamous epithelia the growth 
and renewal is accomplished by what they call progenitor cells (PC), which are 
the stem cells (SC) of the tissue. The PC are located in the basal layer of the 
tissue and they renew the tissue by producing transit amplifying cells (TAC) 
which have limited proliferative potential. These TAC move up through the tissue 
generating new cells and within a few weeks or less are sloughed off. As a 
result of the short lifespan of the TAC they are not good candidates for 
becoming mutated and thus don't create neoplasic lesions. As a result 
\textcite{Ryser} only consider the PC in their microscopic model. Cells go from 
normal to cancerous through accumulation of (epi)genetic aberrations. When a 
normal PC acquires growth advantage its' progeny start spreading through the 
epithelium. Since the number and timing of genetic alterations changes from 
patient to patient \textcite{Ryser} decided to look at phenotypic progression 
instead of genotypic progression. The three histopathological stages 
of epithelial dysplasia (precancerous stages) are mild, moderate and severe 
(carcinoma in situ [CIS]). Based on these three stages \textcite{Ryser} consider 
the following four type of cells: normal cells (type 0), mildly dysplastic cells 
(type 0*), moderately dysplastic cells (type 1), and severely dysplastic cells 
(type 2). They use the stochastic Moran model on a regular two-dimensional 
lattice as described in \textcite{Foo}. \textcite{Ryser} initialize 
the model with normal PCs (type 0) with proliferative rate $f_0$. A type 0 cell 
becomes a type 0* cell at the rate $u_{1, a}$, type 0* becomes type 1 at the 
rate $u_{1, b}$, and type 1 becomes type 2 at the rate $u_2$. Biologically the 
proliferative rate of type 0 and type 0* is the same so type 0* has 
proliferative rate $f_0$. As a result of this the model can be simplifed by 
computing the probability of a type 0* cell becoming a type 1 cell, thus 
resulting in a rate $v_{01}$ that a type 0 cell becomes a type 1 cell. Type 1 
cells have a proliferative advantage over type 0 and type 0* cells due to it 
having a fitness advantage $s_1$ and so its' proliferative rate is given by $f_1 
= f_0(1 + s_1)$. Similiarly the type 2 cells have a proliferative rate of $f_2 = 
f_1(1 + s_2)$. \textcite{Ryser} model the time between onset of CIS and 
diagnosis using an exponentially distributed random varaible with rate $\Psi$. 
To analytically compute the waiting times and field geometries they use the 
mesoscopic approximation to the spatial model from \textcite{Foo}. In this model 
the arrival of expanding type 1 clones is a stochastic Poisson process with rate 
$Nu_1\frac{s_1}{1+s_1}$, where $N$ is the total number of cells. The factor 
$\overline{s_1}=\frac{s_1}{1+s_1}$ represents the idea that progeny of a new 
type 1 cell either go to extinction with probability $1-\overline{s_1}$ or 
expand indefinitely with with probability $\overline{s_1}$. According to a 
theorem, expanding type 1 clones asymptotically grow as a convex symmetric shape 
with constant radial growth rate $c_2$. In \textcite{Foo} it was found in 
particular that the convex symmetric shape cna be approximated as a disk or 
circle. The rate $c_2$ depends on the selective advantage $s_1$. For small $s_1$ 
it scales as $c_s(s_1) \sim \sqrt{4\pi s_1 / \log{1/s_1}}$ where $f(s) \sim 
g(s)$ means that $f(s)/g(s) \rightarrow 1$ as $s \rightarrow 0$. For larger 
values $s_1$ the relationship had to be numerically computed, in particular for 
$s_1 > 0.5$ \textcite{Ryser} found an approximately linear dependence given by 
$c_2(s_1) \approx 0.6s_1 + 0.22$. \textcite{Ryser} allow the existence of 
multiple precancer fields of type 1 cells within the model. To estimate and 
compute the parameters for their model they use age-specific incidence rates 
from the Surveillance, Epidemiology, and End Results (SEER) program of the 
National Cancer Institute (18 registries, 2000-2012) in a Bayesian framework. 
Since \textcite{Ryser} only considered HPV-negative cancers they restricted to 
only HNSCC within the lip, tongue, floor of the mouth, gum and other mouth, 
hypopharynx, and larynx. They computed the number of susceptible individuals and 
the number of cancer cases diagnosed for the following age groups: 15-19, 20-24, 
..., 80-84, and 85+. The final reduction to the data used was to consider only 
smokers within each considered age groups, as tobacco consumption is a major 
cause of HPV-negative HNSCC. \textcite{Ryser} computed the survival function, 
the probability density function of the local field radius, and the probability 
of harboring at least two clonally unrelated fields in the head and neck region 
with respect to the mean age at smoking initiation to diagnosis with invasive 
cancer. They found that there is a strong dependence of the local field size on 
age at diagnosis, with adoublling of the expected field diameter between ages at 
diagnosis of 50 and 90 years. Further the probability of harboring multiple 
clonally unrelated fields at the time of diagnosis were found to increase 
substantially with patient age. As a result of these discoveries 
\textcite{Ryser} suggest that patient age at diagnosis is a critical predictor 
of the size and multiplicity of precancerous lesions. 

% Gerlee review
The next set of literature discusses mathematical techniques that were used 
for the model that will be proposed in this thesis. The first paper by 
\textcite{Gerlee} inspired the general framework of the mathematical model. 
\textcite{Gerlee} created a hybrid cellular automaton to model the effect of 
various microenvironmental factors on solid tumour growth. Their model is a 
hybrid cellular automaton because the rule of the automaton depends upon the 
output of a neural network and partial differential equations. The cellular 
atomaton comprises of two cell types an empty cell (normal cell) and a tumour 
cell. It is initalized by setting all the automaton elements to empty except 
the middle four cells which are occupied by tumour cells. The neural 
network is used to approximate the relationship between the microenvironmental 
variables and the phenotype of a cell. The partial differential equations are 
used to model the spread of the various microenvironmental variables in the 
domain of consideration. For the neural network they use a multi-layer 
perceptron (MLP), with input being the output of the partial differential 
equation for the cell at a location $(x, y)$ and output being a vector of 
likelihoods of a phenotype and movement occuring at a timestep. The hidden 
layer of the MLP represents the genes and hence the neural network attempts to 
replicate the genotype-phenotype relationship. They consider the phenotypes 
proliferation (P), quiescence (Q), and apoptosis (A). Each timestep represents a 
cell cycle so that a single phenotypic action will occur for each cell. The 
maximum of the likelihoods between P, Q, and A determines which phenotypic 
action occurs. If the likelihood of movement is sufficiently large then the cell 
is allowed to move. Each of the actions also has some restrictions based upon 
the cells metabolism and adhesion but these will be ignored due to both of these 
aspects of the cell not being modelled in this thesis. All of the partial 
differential equations are chemical field equations of the form: $\frac{\partial 
c(x, y, t)}{\partial t} = D \Delta c(x, y, t) \pm f(x, y, t)$. Where $c(x, y, 
t)$ is the concentration of the microenvironmental variable, $D$ is the 
diffusion coefficient, $\Delta$ is the laplacian operator, and $$f(x, y, t) 
= \begin{cases} 0, &\text{If the automaton element at } (x, y) \text{ is empty} 
\\ rF(x, y), &\text{If the automaton element is occupied} \end{cases}$$; where 
$r$ is the base consumption/production rates and $F(x, y)$ is the modulated 
energy consumption of the individual cell occupying the automaton element at 
$(x, y)$. The hybrid cellular automaton goes through the following process at 
for each cell and time step:
\begin{enumerate}
  \item The input to the neural network is sampled from the local environment.
  \item The MLP is computed to get the phenotype and likelihood of movement.
  \item The cell consumes nutrients according to the chosen phenotype 
        and the metabolic pathway is chosen.
  \item The chosen phenotype is carried out.
  \item If movement is activated and the cell hasn't divided it tries to move to 
        a neighbouring cell location.
\end{enumerate}

\end{document}