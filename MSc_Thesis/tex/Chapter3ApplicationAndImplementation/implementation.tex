\providecommand{\main}{../..}
\documentclass[\main/thesis.tex]{subfiles}

\begin{document}

\section{Model Implementation}

\subsection{Parallel Implementation}
Due to the vast number of cells that would need to be updated each time-step we utilized GPU parallelization to implement the model resulting in the ability to run the model in real time and vastly speeding up the generation of the results. It has been shown that utilizing asynchronous updates doesn't impact the results of a CA as compared to synchronous updates \cite{Schonfisch}. One problem with using parallelization is that if a cell has been acted upon then it should not itself be able to complete an action due to the fact that its state has been changed. Although, as long as the new state of the cell is not empty, the results are not impacted.

Issues can occur when a cell is searching for another to perform an action upon, due to the mechanism used in our CA allowing the cell to randomly choose a neighbor on which to perform an action. As the cells are updating concurrently, methods had to be developed to prevent multiple neighbours attempting to perform an action upon the same cell. We also had to ensure that while a cell is attempting to perform an action, another cell cannot be attempting to perform an action on that cell. As well we had to ensure that as soon as an action succeeds or fails the search for a new neighbour is ceased. This issue was resolved using a lock mechanism. When a cell is attempting an action, it locks itself to any other cell attempting an action upon it in the same time-step. Similarly, if a cell currently has an action being attempted upon it by another cell, that cell is also locked. When a cell succeeds in an action, that cell is labelled as having completed an action and the cell being acted upon is updated. A cell will search its neighbours until it locates one that is unlocked upon which it attempts an action on that cell. To prevent a cell indefinitely searching it completes the task in a loop until either it succeeds or fails or it reaches a maximum number of searches. The simulation used a maximum value of 100 searches through the neighbours.

\end{document}
