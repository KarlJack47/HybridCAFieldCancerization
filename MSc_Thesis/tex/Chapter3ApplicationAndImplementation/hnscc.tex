\providecommand{\main}{../..}
\documentclass[\main/thesis.tex]{subfiles}

\begin{document}

\section{Application}

\subsection{Introduction to Head and Neck Squamous Cell Carcinoma}

We review head and neck squamous cell carcinoma (HNSCC) as presented in \cite{medlineplus.gov_2021}. Squamous cell carcinoma (SCC) is cancer that originates from squamous cells. These cells are found in the outer layer of skin and in the mucous membranes, which are the moist tissues that line body cavities such as the airways and intestines. There is a particular type of SCC that develops in the mucous membranes of the mouth, nose, and throat, it is called head and neck SCC. HNSCC is classified relative to its location, namely, there are the following main types: oral cavity (occurs in the mouth), oropharynx (middle part of the throat near the mouth), nasal cavity and para-nasal sinuses (space behind the nose), nasopharynx (upper part of the throat near the nasal cavity), larynx (voice-box), and the hypopharynx (lower part of the throat near the larynx). Symptoms of HNSCC include abnormal patches or open sores (ulcers) in the mouth and throat, unusual bleeding or pain in the mouth, sinus congestion that does not clear, sore throat, earache, pain when swallowing or difficulty swallowing, a hoarse voice, difficulty breathing, or enlarged lymph nodes. It can metastasize to other parts of the body such as the lymph nodes or lungs. There is about a 50\% chance of surviving another 5 years after initial diagnosis. HNSCC is the seventh most common type of cancer worldwide with approximately 600,000 new diagnoses each year, including about 50,000 in the US alone. It most often occurs in men in their 50s or 60s. 

Tobacco use, including cigarettes, cigars, pipes, chewing tobacco, and snuff, is the largest risk factor for HNSCC, since it is linked to 85\% of cases \cite{cancer.net_2021}. In the US smoking more than 2 packs of cigarettes per day is the main tobacco-related risk factor for mouth and throat cancer \cite{merckmanuals.com_2021}. Pipe smoking in particular has been linked to cancer in the part of the lips that touch the pipe stem \cite{merckmanuals.com_2021}. Chewing tobacco or snuff is associated with a 50\% increase in the risk of developing cancer in areas of the mouth that comes most in-contact with the tobacco. This includes the cheeks, gums, and inner surface of the lips \cite{merckmanuals.com_2021}. Finally, secondhand smoke may also increase a person's risk of head and neck cancer \cite{merckmanuals.com_2021}. Frequent and heavy alcohol consumption increases the risk of head and neck cancer with the risk increasing proportional to the amount of alcohol a person consumes \cite{merckmanuals.com_2021}. Using alcohol and tobacco together is known to increase the risk of developing head and neck cancer by two to three times more than just one of them alone \cite{merckmanuals.com_2021}. Human papillomavirus (HPV) infection, which induces cancer to develop in the tonsils and base of the tongue, increases the risk of developing throat cancer 16-fold and causes 60\% of throat cancers \cite{merckmanuals.com_2021}. HPV is causing an increase in the number of incidences of HNSCC among younger individuals \cite{merckmanuals.com_2021}. Prolonged sun exposure can increase the risk of cancer in the lip area due to UV radiation \cite{merckmanuals.com_2021}. Gender seems to be a factor in the risk as well since men are more likely to develop HNSCC then women \cite{merckmanuals.com_2021}. People with fairer skin also seem to have an increased risk to develop HNSCC \cite{cancer.net_2021}. Generally people older than 45 have an increased risk for oral cancer, though it can develop in people of any age \cite{cancer.net_2021}. Poor dental care and not following regular oral hygiene practises may cause an increased risk of oral cavity cancer, this risk is even further increased for people that use alcohol and tobacco products \cite{merckmanuals.com_2021}. A diet low in fruits and vegetables and a vitamin A deficiency may increase the risk of oral and oropharyngeal cancer \cite{cancer.net_2021}. Chewing betel nuts, a nut containing a mild stimulant that is popular in Asia, also raises a person's risk of developing oral and oropharyngeal cancer \cite{cancer.net_2021}. It has also been seen that people that have a weakened immune system may have a higher risk of developing HNSCC \cite{cancer.net_2021}. Finally it has been shown that people that use marijuana may be at a higher-than-average risk for HNSCC \cite{cancer.net_2021}. HNSCC is generally not inherited so as result it arises from mutations in the body's cells that occur during an individuals lifetime \cite{medlineplus.gov_2021}. The best ways to prevent the risk of developing HNSCC is to not use any tobacco products and to try to prevent acquiring HPV infection. 

There are many genes related to HNSCC but the top nine genes are: ING1, PTEN, TNFRSF10B, TP53, MIR21, MIR210, MIR205, MIR98, and ING3 \cite{malacards.org_2021}. Looking specifically at HNSCC in the tongue the top twenty genes that are related are: TP53, FAT1, CDKN2A, NOTCH1, PIK3CA, KMT2D, FAT4, CASP8, MYH9, EP300, NSD1, HRAS, NOTCH2, MLLT4, FBXW7, NFE2L2, AKAP9, GRIN2A, RB1, and CDH11 \cite{malacards.org_2021}. 

The reason we use HNSCC for a case study of the model proposed in this thesis is twofold, first, it is the first type of cancer that field cancerization was discovered, and secondly, it is a widely studied case. This thesis will consider the case of tobacco and alcohol as carcinogens as related to HNSCC of the tongue. 

\end{document}