\providecommand{\main}{../..}
\documentclass[\main/thesis.tex]{subfiles}

\begin{document}

\section{Cellular Automaton}
As mentioned in the model overview, the six cell classes that we consider are 
normal tissue cells (NTC), mutated normal tissue cells (MNTC), normal stem cells 
(NSC), mutated normal stem cells (MNSC), cancer stem cells (CSC), and tumour 
cells (TC). Each cell class is specified in the cellular automata (CA) using a 
numerical value between 0 and 5. More precisely, we have that \newline
$0 {=} \text{NTC}$, $1 {=} \text{MNTC}$, $2 {=} \text{NSC}$, $3 {=} \text{MNSC}$, $4 {=} \text{CSC}$, $5 {=} \text{TC}$.
Since biological cells can move, proliferate, differentiate, and go through 
apoptosis then we must also introduce an empty cell class which is represented 
by the value 6. The cell class in the CA is represented by 
$s(\boldsymbol{x}, t) {\in} \{ 0, 1, ..., 6  \}$. Note that when visualizing the CA each 
value of $s(\boldsymbol{x}, t)$ also has a colour associated to it. 
Though we can choose any arbitrary colour for each cell class the colours chosen for the 
results of the model can be seen in table \ref{table:CAStates}.
\begin{table}[H]
\centering	
\begin{tabular}{|c|c|c|}
	\hline
	Cell class & $s(\boldsymbol{x}, t)$ & Colour \\
	\hline
	normal tissue cell (NTC) & 0 & \cellcolor[RGB]{146, 111, 98} brown \\
	mutated normal tissue cell (MNTC) & 1 & \cellcolor[RGB]{50, 200, 118} green \\
	normal stem cell (NSC) & 2 & \cellcolor[RGB]{0, 84, 147} blue \\
	mutated normal stem cell (MNSC) & 3 & \cellcolor[RGB]{240, 200, 0} yellow \\
	cancer stem cell (CSC) & 4 & \cellcolor[RGB]{200, 62, 255} purple \\
	tumour cell (TC) & 5 & \cellcolor[RGB]{255, 61, 62} red \\
	empty cell & 6 & \cellcolor[RGB]{255, 255, 255} white \\
	\hline
\end{tabular}
\caption{CA cell classes}
\label{table:CAStates} 
\end{table}

Each cell in the CA tracks the gene expression of the $G$ genes in a vector defined by
\begin{equation}
    \boldsymbol{E}(\boldsymbol{x}, t) {=} [ \{ e_j(\boldsymbol{x}, t) \}_{j{=}1, ..., G} ].
    \label{eq:genExprsVector}
\end{equation}
The phenotype of a cell is tracked by a vector that contains probabilities for each type of phenotypic action occurring in a given time-step and is defined by
\begin{equation}
\boldsymbol{P}(\boldsymbol{x}, t) {=} [ p(\boldsymbol{x}, t), q(\boldsymbol{x}, t), a(\boldsymbol{x}, t), d(\boldsymbol{x}, t) ],
\label{eq:phenotypeVector}
\end{equation}
where $p(\boldsymbol{x}, t)$ represents proliferation, $q(\boldsymbol{x}, t)$ represents quiescence, $a(\boldsymbol{x}, t)$ represents apoptosis, and $d(\boldsymbol{x}, t)$ represents differentiation. The probabilities are set such that $\boldsymbol{P}(\boldsymbol{x}, t)$ generates a probability distribution, so that 
\begin{equation}
\sum_{i{=}1}^4 P_i(\boldsymbol{x}, t) {\equiv} p(\boldsymbol{x}, t) {+} q(\boldsymbol{x}, t) {+} a(\boldsymbol{x}, t) {+} d(\boldsymbol{x}, t) {=} 1 \text{ and } P_i(\boldsymbol{x}, t) {\ge} 0, {\forall} t.
\label{eq:phenotypeSumCondition}
\end{equation}
At a time-step in the CA a phenotypic action is chosen to occur by sampling from the probability distribution generated from $\boldsymbol{P}(\boldsymbol{x}, t)$. Hence, since we do not want a cell to reproduce more than once in a time-step, each time-step represents the length of a typical cell cycle for the type of tissue under consideration.

When a NSC, MNSC, or CSC differentiate the resultant cell 
initially is a transient amplifying cell (TAC) for a set number of generations, 
$\Theta$, after which it turns respectively into a NTC, MNTC, or TC. As a result 
of this each cell has two parameters $\overline{\tau}(\boldsymbol{x}, t) {\in} \{0, 1\}$ and
$\overline{n}(\boldsymbol{x}, t) {\in} \{0, ..., \Theta\}$, where 
$\overline{\tau}(\boldsymbol{x}, t)$ is a binary parameter used to determine if a cell is currently 
a TAC or not and $\overline{n}(\boldsymbol{x}, t)$ is the number of generations a TAC cell lineage has produced. The parameters $\overline{\tau}(\boldsymbol{x}, t)$ and $\overline{n}(\boldsymbol{x}, t)$ are copied from parent to child cell and once $\overline{n}(\boldsymbol{x}, t) {=} \Theta$ then 
$\overline{\tau}(\boldsymbol{x}, t{+}1) {=} 0$, $\overline{n}(\boldsymbol{x}, t{+}1) {=} 0$. 

The final aspect of the cell that is tracked and represented in the overall cell state is the age of the cell, $\alpha(\boldsymbol{x}, t) {\in} \mathbb{N}$. The state of a cell in the CA is given by the vector
\begin{align}
\boldsymbol{S}(\boldsymbol{x}, t) {=} [ s(\boldsymbol{x}, t), 
                                        \alpha(\boldsymbol{x}, t),
                                        \boldsymbol{E}(\boldsymbol{x}, t),
                                        \boldsymbol{P}(\boldsymbol{x}, t),
                                        \overline{\tau}(\boldsymbol{x}, t),
                                        \overline{n}(\boldsymbol{x}, t) ]
\label{eq:StateVector}
\end{align}

The domain could theoretically be either two-dimensional or \newline three-dimensional, however for simplicity and computational purposes, the domain considered will be two-dimensional, so $\boldsymbol{x} {\in} \mathbb{R}^2$. Each cell has a neighbourhood that contains itself, the cardinal directions 
around it, and the cells directly NE, SE, SW, and NW of the cell. In CA theory this is called the Moore neighbourhood and is mathematically defined for two-dimensional grids as
\begin{equation}
N^M_{(x_0, y_0)} {=} \{ (x, y) | |x {-} x_0| {\le} r, |y {-} y_0| {\le} r \},
\label{eq:MooreNeighbourhood}
\end{equation}
where $r$ is the range of the Moore neighbourhood and $(x_0, y_0)$ is the cell that the neighbourhood 
surrounds \cite{Gray}. Most commonly the Moore neighbourhood used is with $r {=} 1$ and it will be the one used in this thesis. 

The boundary conditions of the grid are standard periodic boundary conditions. 
Periodic boundaries were chosen because they are easier to work with and the size of the domain physically speaking is minuscule so they will not effect the results. We could easily also use non-periodic boundaries by dealing with edge cases.

\subsection{Cell Mutation}
A cell can become mutated through changes in both the likelihood of a phenotype 
occurring and the gene expression. The process of these changes is described in 
this section.

\subsubsection{How Mutations Effect Gene Expression}
The chosen $G$ genes are known genes related to the type of cancer being studied.
Thus, each of the $G$ genes is either a tumour suppressor gene or an oncogene.
Define the vector $\boldsymbol{T} {\in} \{ 0, 1 \}^G$, where each element, 
$T_{j}$, represents gene $j$'s type, whereby $T_{j} {=} 0$ 
represents a tumour suppressor gene and $T_{j} {=} 1$ represents an oncogene.

A gene $j$ is positively mutated towards cancer (positively mutated) if it is mutated and either it is a tumour suppressor gene and its gene expression is downregulated, $T_j {=} 0$ and $e_j(\boldsymbol{x}, t) {\le} -\overline{M}$, or it is an oncogene and its gene expression is upregulated, $T_j {=} 1$ and $e_j(\boldsymbol{x}, t) {\ge} \overline{M}$, where $\overline{M}$ is the given threshold.

At each time-step the gene expression of each gene is updated from the results 
of the gene expression neural network. The changes in the gene expression allow 
the gene to become mutated or even go from mutated to non mutated (normally 
expressed). The following function is used as an indicator to determine if a gene $j$ is mutated, positively mutated or normally expressed:
\begin{equation}
\Psi(\boldsymbol{x}, t, j) {=} \begin{cases}
                             0, & |e_j(\boldsymbol{x}, t)| {<} \overline{M} 
                               (\text{{\footnotesize normally expressed}})\\
                             1 {-} 2T_j, & e_j(\boldsymbol{x}, t) {\le} {-}\overline{M}
                               (\text{{\footnotesize underexpressed gene}}) \\
                             {-}2 {+} 4T_j, & e_j(\boldsymbol{x}, t) {\ge} \overline{M} 
                               (\text{{\footnotesize overexpressed gene}})
                        \end{cases}
\label{eq:MutationFunc}
\end{equation}
Notice that when $\Psi(\boldsymbol{x}, t, j) {>} 0$ the gene $j$ is positively mutated and when $\Psi(\boldsymbol{x}, t, j) {<} 0$ it is mutated away from cancer. Further take note that if $$|\Psi(\boldsymbol{x}, t, j)| {=} 1$$ the gene $j$ is downregulated and when $$|\Psi(\boldsymbol{x}, t, j)| {=} 2$$ it is upregulated. 

\subsubsection{Update Rules for Gene Expression}
A given gene can influence the expression of another gene as follows. 
A positively mutated gene will cause a positive mutation of a related gene. 
A non-positively mutated gene will cause a negative mutation (mutation that regulates a gene towards normal expression) of a related gene. 
Define the matrix $R {\in} \{ 0, 1 \}^{G{\times}G}$, where each entry, $R_{ij}$, 
represents whether gene $i$ is related to gene $j$ with $0{=}\text{unrelated}$ 
and $1{=}\text{related}$. Note that the matrix $R$ is not necessarily symmetric as a 
gene $i$ might regulate gene $j$ but not vice versa. To  update gene $j$ according to formula \eqref{eq:geneInstabilityUpdateRule} below, we chose a random number $z {\sim} U(0,1)$ and an update occurs only if $z {\leq} \overline{\gamma}$ and $R_{ij} {=} 1$.
The process of a gene $i$ changing the gene expression of another gene $j$ is 
represented by the formula:
\begin{align}
&\varsigma(\boldsymbol{x}, t, i, j) {=}
\begin{cases}
   1, &(\Psi(\boldsymbol{x}, t, i) {>} 0 \text{ and } T_j {=} 1 
       \text{ and } e_j(\boldsymbol{x}, t) {<} M) \text{ or }\\
      &(\Psi(\boldsymbol{x}, t, i) {\le} 0 \text{ and } e_j(\boldsymbol{x}, t) {<} 0)\\
  \\
  -1, &(\Psi(\boldsymbol{x}, t, i) {>} 0 \text{ and } T_j {=} 0 
       \text{ and } e_j(\boldsymbol{x}, t) {>} {-}M) \text{ or }\\
      &(\Psi(\boldsymbol{x}, t, i) {\le} 0 \text{ and } e_j(\boldsymbol{x}, t) {>} 0)\\
  \\
  0, &\text{otherwise}
\end{cases}, \label{eq:geneInstabilityUpdateFunction}\\
&e_j(\boldsymbol{x}, t) {=} e_j(\boldsymbol{x}, t{-}1) {+} \varsigma(\boldsymbol{x}, t{-}1, i, j) \varepsilon,
\label{eq:geneInstabilityUpdateRule}
\end{align}
where $\varepsilon {\sim} U(\varepsilon_1,\varepsilon_2)$ is a randomly chosen increment between $\varepsilon_1$ and $\varepsilon_2$.  The previous update is used by fixing a 
gene $i$ and applying the function to all the other genes, then repeating the process on the next gene and so forth, until all the genes have been processed. In each of these steps new random numbers $z,\varepsilon$ are generated. 

In addition if a gene $j$ is mutated then there is a chance that the gene 
expression is negatively mutated, so to replicate the bodies
attempting to revert mutated genes. The following function is applied only when 
some random variable $z {\sim} U(0,1)$ is less than or equal to a threshold $\overline{\phi}$.
This is represented by the gene repair function:
\begin{align}
e_j(\boldsymbol{x}, t) {=} e_j(\boldsymbol{x}, t{-}1) {+}
\begin{cases}
    \varepsilon, 
      & |\Psi(\boldsymbol{x}, t{-}1, j)| {=} 1 \\
    {-} \varepsilon, 
      & |\Psi(\boldsymbol{x}, t{-}1, j)| {=} 2 \\
\end{cases},
\label{eq:geneExpressionFixer}
\end{align}
where $\varepsilon {\sim} U(\varepsilon_1, \varepsilon_2)$ is a random increment.  
In a given time-step the previous function \eqref{eq:geneExpressionFixer} is applied once to all $G$ genes.

\subsubsection{Update Rules for Phenotypic Action}
The probability of a phenotypic action occurring can change at each time-step based upon the gene expression of a gene. When a gene is mutated it can modify the probability of a phenotypic action occurring, $P_i(\boldsymbol{x}, t)$.
We define the matrix $\overline{U} {\in} \mathbb{R}^{4{\times}G}$, where each entry, 
$\overline{U}_{ij}$, is an increment to the probability of phenotypic action $i$,
$P_i(\boldsymbol{x}, t)$, under the circumstance that gene $j$ is mutated and its' 
expression is upregulated. Similarly, we define the matrix
$\overline{D} {\in} \mathbb{R}^{4{\times}G}$, where each entry, $\overline{D}_{ij}$, is 
an increment to the probability of phenotype action $i$, $P_i(\boldsymbol{x}, t)$, under the circumstance that gene $j$ is mutated and its' expression is downregulated. We define updates for the phenotypic actions in \eqref{eq:phenotypeMutate} and \eqref{eq:phenotypeMutateBalance} in a way such that  the probability of a phenotypic action occurring is kept bounded between $0$ and $1$ and
$\sum_{i{=}1}^{4} P_i(\boldsymbol{x}, t) {=} 1$. 

The sum of the phenotype vector equaling one is maintained by balancing the probability of 
each phenotype action against the probability of quiescence and quiescence equally against all the other phenotypic actions. Note that usually $\overline{U}{=}{-}\overline{D}$ since the effects of the up-regulation of a gene $j$ has the opposite effect on the phenotypic action $i$ relative to the down-regulation of the gene. For each update we choose a random sample $\varpi {\sim} U(0, 1)$ which denotes the magnitude of the change to $P_i$ based on a change in $e_j$. To ensure that updates remain between $0$ and $1$ we use the following modified expression for the increment:
\begin{equation}
\Xi(\epsilon, \xi) {=} \begin{cases}
                          {-}\xi {\times} 0.99,
                           & \xi {+} \epsilon {\le} 0 \\
                          (1 {-} \xi) {\times} 0.99,
                           & \xi {+} \epsilon {\ge} 1 \\
                          \epsilon, & \text{otherwise}
                         \end{cases},
\label{eq:phenotypicActionBoundaryFunction}
\end{equation}
Given the mutated gene $j$ we change the phenotypic probability $P_i$ with the following formulas, where $\Phi$ defines the maximum increment or detriment amount:
\begin{equation}
\Phi(\boldsymbol{x}, t, i) =
\begin{cases}
  \overline{D}_{ij}, &|\Psi(\boldsymbol{x}, t, j)| = 1 \\
  \overline{U}_{ij}, &|\Psi(\boldsymbol{x}, t, j)| = 2
\end{cases}
\label{eq:phenotypeMutateEpsilon}
\end{equation}
Then the update for $P_i$ ensures that the new probabilities remain between $0$ and $1$. 
\begin{align}
P_i(\boldsymbol{x}, t) &{=} P_i(\boldsymbol{x}, t{-}1) \nonumber \\ &{+}
                           \begin{cases}
                             \text{sgn}(\Phi(\boldsymbol{x}, t, i))
                             \min(|\Xi(\varpi \Phi(\boldsymbol{x}, t, i),  
                                   P_i(\boldsymbol{x}, t{-}1))|,   
                               & , i {\ne} 2 \\
                             \hspace{3.4cm} |\Xi({-}\varpi \Phi(\boldsymbol{x}, t, i),  
                                             P_2(\boldsymbol{x}, t{-}1))|)\\
                             \\
                             \Xi(\varpi \Phi(\boldsymbol{x}, t, i), 
                                 P_2(\boldsymbol{x}, t{-}1))
                               & , i {=} 2
                            \end{cases},
\label{eq:phenotypeMutate}
\end{align}
These changes are balanced with changes to the other phenotype actions $k {\neq} i$ to ensure full probability. 
\begin{align}
P_k(\boldsymbol{x}, t) &{=} P_k(\boldsymbol{x}, t{-}1) \nonumber \\ &{-}
                           \begin{cases}
                              \text{sgn}(\Phi(\boldsymbol{x}, t, i))
                              \min(|\Xi(\varpi \Phi(\boldsymbol{x}, t, i), 
                                    P_i(\boldsymbol{x}, t{-}1))|,
                                & , k {=} 2 \\
                              \hspace{3.4cm} |\Xi({-}\varpi \Phi(\boldsymbol{x}, t, i),
                                              P_k(\boldsymbol{x}, t{-}1))|)
                             \\
                             \\
                             \displaystyle \frac{\Xi(\varpi \Phi(\boldsymbol{x}, t, i),
                             	                     P_i(\boldsymbol{x}, t{-}1))
                             	                 P_k(\boldsymbol{x}, t{-}1)}
                                                {\sum_{l{\ne}2} P_l(\boldsymbol{x}, t{-}1)}
                               & , k {\ne} 2
                            \end{cases}, 
\label{eq:phenotypeMutateBalance}
\end{align}
The equation \eqref{eq:phenotypeMutate} modifies a phenotypic action with 
respect to a gene $j$, with equation \eqref{eq:phenotypeMutateBalance} balancing those changes so that the sum of the probability distribution generated by $\boldsymbol{P}(\boldsymbol{x}, t)$ remains one. Thus, to mutate a phenotypic action $i$ with respect to gene $j$, we use equation \eqref{eq:phenotypeMutate} then if $i {\ne} 2$, we are not modifying quiescence, and we use equation \eqref{eq:phenotypeMutateBalance} with $k {=} 2$, otherwise we use equation \eqref{eq:phenotypeMutateBalance} with $k {=} 1, 3, 4$. 
Since $|\Psi(\boldsymbol{x}, t, j)| {=} 1$ means that gene $j$ is downregulated, then we know that in that case the increment for the phenotypic action $i$ with respect to gene $j$ will be $D_{ij}$. 
Similarly, whenever $|\Psi(\boldsymbol{x}, t, j)| {=} 2$ then the increment for the phenotypic action $i$ will be $U_{ij}$. Mathematically speaking this update process is
accomplished through a composition of applications of \eqref{eq:phenotypeMutate} and 
\eqref{eq:phenotypeMutateBalance}.
We define the functional $\overline{P}^j(\boldsymbol{x}, t, i)$ where $i$ is the 
phenotypic action being modified using \eqref{eq:phenotypeMutate} and 
\eqref{eq:phenotypeMutateBalance} with respect to gene $j$. Thus the update can be written as 
$P_i(\boldsymbol{x}, t) = \overline{P}^1(\overline{P}^2(...\overline{P}^G(\boldsymbol{x}, t, i)))$, where in each step we re-sample random variables $z,\varpi$ as described above. 

\begin{lemma}
Let $P_l(\boldsymbol{x}, t{-}1) {\in} (0, 1), l {=} 1, 2, 3, 4$ and 
$\sum_{l{=}1}^4 P_l(\boldsymbol{x}, t{-}1) {=} 1$. If the probability $P_i$ of phenotypic action $i\in\{1,2,3,4\}$ is modified 
using the rules \eqref{eq:phenotypeMutate} and \eqref{eq:phenotypeMutateBalance} 
with respect to gene $j$ then we will still have $\sum_{l{=}1}^4 P_l(\boldsymbol{x}, t) {=} 1$, 
$\forall t {>} 0$.
\end{lemma}
\begin{proof}
	Assume $\Psi(\boldsymbol{x}, t, j) {\ne} 0$ and the threshold for gene $j$ to modify the probability of 
	phenotypic action $i$ is met, \ie, $z {\sim} Z(0, 1) {\le} \overline{\rho}$. Also since the increment 
	being applied to the probability of phenotypic action $i$ has only two possible values related to whether 
	$|\Psi(\boldsymbol{x}, t, j)| {=} 1$ or $|\Psi(\boldsymbol{x}, t, j)| {=} 2$, then we can assume without 
	loss of generality that the increment is some value $\epsilon$. If $\epsilon {=} 0$ then trivially we 
	have that $\sum_{l{=}1}^4 P_l(\boldsymbol{x}, t) {=} 1$ since no changes will occur to 
	$P_i(\boldsymbol{x}, t{-}1)$.
	\begin{itemize}
	  \item[{\bf Case 1:}] First we assume that $i {\ne} 2$ and $k {=} 2$ as we are balancing the change in 
	                       the probability of phenotypic action $i$ with the probability of quiescence.
	                       Further assume that $P_l(\boldsymbol{x}, t) {=} P_l(\boldsymbol{x}, t{-}1)$, 
	                       $l {\ne} i, 2$.
	                       \begin{itemize}
	                       	 \item[{\bf Case A:}] Assume $\epsilon {<} 0$.
                             \begin{itemize}
                           	   \item[{\bf Case a:}] Assume $0 {<} P_i(\boldsymbol{x}, t{-}1) {+} \epsilon {<} 1$ 
                           	                        and $0 {<} P_2(\boldsymbol{x}, t{-}1) {-} \epsilon {<} 1$. 
                           	                        We have that 
                           	                        \begin{align*}
                           	                          &\text{sgn}(\epsilon) 
                           	                           \min(|\Xi(\epsilon, 
                           	                                 P_i(\boldsymbol{x}, t{-}1))|, 
                           	                                |\Xi({-}\epsilon,
                           	                                 P_2(\boldsymbol{x}, t{-}1))|)\\
                           	                          &{=} \text{sgn}(\epsilon) 
                           	                                \min(|\epsilon|,
                           	                                     |{-}\epsilon|)
                           	                          {=} \text{sgn}(\epsilon) 
                           	                               |\epsilon| {=} \epsilon.
                           	                          \tag{1}
                           	                          \label{eq:incr1Aa}
                           	                        \end{align*}
                           	                        Using equation (\ref{eq:phenotypeMutate}) and considering 
                           	                        (\ref{eq:incr1Aa}) we have that
                           	                        \begin{align*}
                           	                          P_i(\boldsymbol{x}, t) {=} P_i(\boldsymbol{x}, t{-}1) {+} \epsilon.
                           	                          \tag{2}
                           	                          \label{eq:Pi1Aa}
                           	                        \end{align*}
                           	                        Also using equation (\ref{eq:phenotypeMutateBalance}) and considering 
                           	                        again (\ref{eq:incr1Aa}) we have that
                           	                        \begin{align*}
                           	                          P_2(\boldsymbol{x}, t) {=} P_2(\boldsymbol{x}, t{-}1) {-} \epsilon.
                           	                          \tag{3}
                           	                          \label{eq:P21Aa}
                           	                        \end{align*}
                           	                        Therefore, using equations (\ref{eq:Pi1Aa}) and (\ref{eq:P21Aa}) we have 
                           	                        \begin{align*}
                           	                          \sum_{l{=}1}^4 P_l(\boldsymbol{x}, t) 
                           	                          &{=} \underset{l {\ne} i, 2}{\sum_{l{=}1}^4} P_l(\boldsymbol{x}, t)
                           	                           {+} (P_i(\boldsymbol{x}, t{-}1) {+} \epsilon)
                           	                           {+} (P_2(\boldsymbol{x}, t{-}1) {-} \epsilon) \\
                           	                          &{=} \underset{l {\ne} i, 2}{\sum_{l{=}1}^4} P_l(\boldsymbol{x}, t)
                           	                           {+} P_i(\boldsymbol{x}, t{-}1) {+} P_2(\boldsymbol{x}, t{-}1) \\
                           	                          &{=} \underset{l {\ne} i, 2}{\sum_{l{=}1}^4} P_l(\boldsymbol{x},t{-}1) 
                           	                           {+} P_i(\boldsymbol{x}, t{-}1) {+} P_2(\boldsymbol{x}, t{-}1) \\
                           	                          &{=} \sum_{l{=}1}^4 P_l(\boldsymbol{x}, t{-}1) {=} 1.
                           	                        \end{align*}
                               \item[{\bf Case b:}] Assume $P_i(\boldsymbol{x}, t{-}1) {+} \epsilon {\le} 0$ 
                           	                        and $0 {<} P_2(\boldsymbol{x}, t{-}1) {-} \epsilon {<} 1$. Since\\
                           	                        $P_i(\boldsymbol{x}, t{-}1) {\le} {-}\epsilon$ we have 
                           	                        \begin{align*}
                           	                          &\text{sgn}(\epsilon) \min(|\Xi(\epsilon, P_i(\boldsymbol{x},t{-}1))|, 
                           	                                                     |\Xi({-}\epsilon, 
                           	                                                          P_2(\boldsymbol{x},t{-}1))|)
                           	                          \\ 
                           	                          &{=} {-} \min(|{-}0.99 P_i(\boldsymbol{x}, t{-}1)|, |{-}\epsilon|) \\
                           	                          &{=} {-} \min(0.99 P_i(\boldsymbol{x}, t{-}1), {-}\epsilon) 
                           	                           {=} {-} 0.99 P_i(\boldsymbol{x}, t{-}1).
                           	                          \tag{4}
                           	                          \label{eq:incr1Ab}
                           	                        \end{align*}
                           	                        Using equation (\ref{eq:phenotypeMutate}) and considering 
                           	                        (\ref{eq:incr1Ab}) we have that
                                                    \begin{align*}
                           	                          P_i(\boldsymbol{x}, t) {=} P_i(\boldsymbol{x}, t{-}1)
                           	                                                 {-} 0.99 P_i(\boldsymbol{x}, t{-}1)
                           	                                                 {=} 0.01 P_i(\boldsymbol{x}, t{-} 1).
                           	                          \tag{5}
                           	                          \label{eq:Pi1Ab}
                           	                        \end{align*}
                           	                        Also using equation (\ref{eq:phenotypeMutateBalance}) and considering 
                           	                        again (\ref{eq:incr1Ab}) we have that
                           	                        \begin{align*}
                           	                          P_2(\boldsymbol{x}, t) {=} P_2(\boldsymbol{x}, t{-}1) 
                           	                                                 {+} 0.99 P_i(\boldsymbol{x}, t{-}1).
                                                      \tag{6}
                           	                          \label{eq:P21Ab}
                           	                        \end{align*}
                           	                        Therefore, using equations (\ref{eq:Pi1Ab}) and (\ref{eq:P21Ab}) we 
                           	                        have 
                           	                        \begin{align*}
                           	                          \sum_{l{=}1}^4 P_l(\boldsymbol{x}, t) 
                           	                          &{=} \underset{l {\ne} i, 2}{\sum_{l{=}1}^4} P_l(\boldsymbol{x}, t)
                           	                           {+} 0.01 P_i(\boldsymbol{x}, t{-} 1)\\
                           	                          &{+} (P_2(\boldsymbol{x}, t{-}1) {+} 0.99 P_i(\boldsymbol{x}, t{-}1))
                           	                          \\
                           	                          &{=} \underset{l {\ne} i, 2}{\sum_{l{=}1}^4} P_l(\boldsymbol{x}, t)
                           	                           {+} P_i(\boldsymbol{x}, t{-}1) {+} P_2(\boldsymbol{x}, t{-}1) \\
                           	                          &{=} \underset{l {\ne} i, 2}{\sum_{l{=}1}^4} P_l(\boldsymbol{x},t{-}1) 
                           	                           {+} P_i(\boldsymbol{x}, t{-}1) {+} P_2(\boldsymbol{x}, t{-}1) \\
                           	                          &{=} \sum_{l{=}1}^4 P_l(\boldsymbol{x}, t{-}1) {=} 1.
                           	                        \end{align*}
                               \item[{\bf Case c:}] Assume $0 {<} P_i(\boldsymbol{x}, t{-}1) {+} \epsilon {<} 1$ 
                           	                        and $P_2(\boldsymbol{x}, t{-}1) {-} \epsilon {\ge} 1$. Since\\
                           	                        $1 {-} P_2(\boldsymbol{x}, t{-}1) {\le} {-}\epsilon$ we have 
                                                 	\begin{align*}
                           	                          &\text{sgn}(\epsilon) \min(|\Xi(\epsilon,P_i(\boldsymbol{x},t{-}1))|, 
                           	                                                     |\Xi({-}\epsilon, 
                           	                                                          P_2(\boldsymbol{x},t{-}1))|)
                           	                          \\ 
                           	                          &{=} {-}\min(|\epsilon|, |0.99 (1{-}P_2(\boldsymbol{x}, t{-}1))|) \\
                           	                          &{=} {-} \min({-}\epsilon, 0.99 (1{-}P_2(\boldsymbol{x}, t{-}1))) 
                           	                           {=} 0.99 (P_2(\boldsymbol{x}, t{-}1) {-} 1).
                           	                          \tag{7}
                           	                          \label{eq:incr1Ac}
                           	                        \end{align*}
                           	                        Using equation (\ref{eq:phenotypeMutate}) and considering 
                           	                        (\ref{eq:incr1Ac}) we have that
                           	                        \begin{align*}
                           	                          P_i(\boldsymbol{x}, t) &{=} P_i(\boldsymbol{x}, t{-}1)
                           	                                                  {+} 0.99(P_2(\boldsymbol{x}, t{-}1) {-} 1) \\
                           	                                                 &{=} P_i(\boldsymbol{x}, t{-}1)
                           	                                                  {+} 0.99 P_2(\boldsymbol{x}, t{-}1)
                           	                                                  {-} 0.99.
                           	                          \tag{8}
                           	                          \label{eq:Pi1Ac}
                           	                        \end{align*}
                           	                        Also using equation (\ref{eq:phenotypeMutateBalance}) and considering 
                           	                        again (\ref{eq:incr1Ac}) we have that
                           	                        \begin{align*}
                           	                          P_2(\boldsymbol{x}, t) &{=} P_2(\boldsymbol{x}, t{-}1) 
                           	                                                  {-} 0.99 (P_2(\boldsymbol{x}, t{-}1) {-} 1) \\
                           	                                                 &{=} 0.01 P_2(\boldsymbol{x}, t{-}1)
                           	                                                  {+} 0.99.
                           	                          \tag{9}
                           	                          \label{eq:P21Ac}
                           	                        \end{align*}
                           	                        Therefore, using equations (\ref{eq:Pi1Ac}) and (\ref{eq:P21Ac}) we 
                           	                        have 
                           	                        \begin{align*}
                           	                          \sum_{l{=}1}^4 P_l(\boldsymbol{x}, t) 
                           	                          &{=} \underset{l {\ne} i, 2}{\sum_{l{=}1}^4} P_l(\boldsymbol{x}, t)
                           	                           {+} (P_i(\boldsymbol{x}, t{-} 1) {+} 0.99 P_2(\boldsymbol{x}, t{-}1) 
                           	                           {-} 0.99) \\
                           	                          &\hspace{0.25cm} {+} (0.01 P_2(\boldsymbol{x}, t{-}1) {+} 0.99) \\
                           	                          &{=} \underset{l {\ne} i, 2}{\sum_{l{=}1}^4} P_l(\boldsymbol{x}, t)
                           	                           {+} P_i(\boldsymbol{x}, t{-}1) {+} P_2(\boldsymbol{x}, t{-}1) \\
                           	                          &{=} \underset{l {\ne} i, 2}{\sum_{l{=}1}^4}P_l(\boldsymbol{x},t{-}1) 
                           	                           {+} P_i(\boldsymbol{x}, t{-}1) {+} P_2(\boldsymbol{x}, t{-}1) \\
                           	                          &{=} \sum_{l{=}1}^4 P_l(\boldsymbol{x}, t{-}1) {=} 1.
                           	                        \end{align*}
                               \item[{\bf Case d:}] Assume $P_i(\boldsymbol{x}, t{-}1) {+} \epsilon {\le} 0$ 
                           	                        and $P_2(\boldsymbol{x}, t{-}1) {-} \epsilon {\ge} 1$. 
                           	                        We compute 
                           	                        \begin{align*}
                           	                          &\text{sgn}(\epsilon) \min(|\Xi(\epsilon, P_i(\boldsymbol{x},t{-}1))|, 
                                         	                                     |\Xi({-}\epsilon,
                                         	                                          P_2(\boldsymbol{x}, t{-}1))|)
                           	                          \\ 
                           	                          &{=} {\minus}\min(|{\minus}0.99 P_i(\boldsymbol{x}, t{-}1)|,
                           	                                       |0.99 (1{-}P_2(\boldsymbol{x}, t{-}1))|) \\
                           	                          &{=} {\minus} \min(0.99 P_i(\boldsymbol{x}, t{-}1),
                           	                                        0.99 (1{-}P_2(\boldsymbol{x}, t{-}1))).
                           	                          \tag{10}
                           	                          \label{eq:incr1Ad}
                           	                        \end{align*}
                           	                        There are two possible situations for the minimum in equation 
                           	                        (\ref{eq:incr1Ad}) but both are similar respectively to Case b and 
                           	                        c and so in either situation 
                           	                        $\sum_{l{=}0}^4 P_l(\boldsymbol{x}, t) {=} 1$.
                             \end{itemize}
                             \item[{\bf Case B:}] Assume $\epsilon {>} 0$.
                             \begin{itemize}
                             	\item[{\bf Case a:}] Assume $0 {<} P_i(\boldsymbol{x}, t{-}1) {+} \epsilon {<} 1$ 
                             	                     and $0 {<} P_2(\boldsymbol{x}, t{-}1) {-} \epsilon {<} 1$. 
                             	                     Same as Case (1, A, a) so $\sum_{l{=}1}^4 P_l(\boldsymbol{x}, t) {=} 1$.
                             	\item[{\bf Case b:}] Assume $P_i(\boldsymbol{x}, t{-}1) {+} \epsilon {\ge} 1$ 
                             	                     and $0 {<} P_2(\boldsymbol{x}, t{-}1) {-} \epsilon {<} 1$. Since\\
                             	                     $1 {-} P_i(\boldsymbol{x}, t{-}1) {\le} \epsilon$ we have 
                             	                     \begin{align*}
                             	                       &\text{sgn}(\epsilon) \min(|\Xi(\epsilon,P_i(\boldsymbol{x},t{-}1))|, 
                             	                                                  |\Xi({-}\epsilon, 
                             	                                                       P_2(\boldsymbol{x},t{-}1))|) \\ 
                             	                       &{=} \min(|0.99 (1{-}P_i(\boldsymbol{x}, t{-}1))|, |\epsilon|) \\
                                             	       &{=} \min(0.99 (1{-}P_i(\boldsymbol{x}, t{-}1)), \epsilon) 
                            	                        {=} 0.99 (1 {-} P_i(\boldsymbol{x}, t{-}1)).
                             	                       \tag{11}
                             	                       \label{eq:incr1Bb}
                             	                     \end{align*}
                             	                     Using equation (\ref{eq:phenotypeMutate}) and considering 
                             	                     (\ref{eq:incr1Bb}) we have that
                             	                     \begin{align*}
                             	                       P_i(\boldsymbol{x}, t) &{=}
                             	                       P_i(\boldsymbol{x}, t{-}1)
                             	                       {+} 0.99 (1 {-} P_i(\boldsymbol{x}, t{-}1))\\
                             	                       &{=} 0.01 P_i(\boldsymbol{x}, t{-} 1) {+} 0.99.
                             	                       \tag{12}
                             	                       \label{eq:Pi1Bb}
                             	                     \end{align*}
                             	                     Also using equation (\ref{eq:phenotypeMutateBalance}) and considering 
                             	                     again (\ref{eq:incr1Bb}) we have that
                             	                     \begin{align*}
                             	                       P_2(\boldsymbol{x}, t) &{=} P_2(\boldsymbol{x}, t{-}1) 
                             	                                               {-} 0.99 (1{-}P_i(\boldsymbol{x}, t{-}1))\\
                             	                                              &{=} P_2(\boldsymbol{x}, t{-}1)
                             	                                               {+} 0.99 P_i(\boldsymbol{x}) {-} 0.99.
                             	                       \tag{13}
                             	                       \label{eq:P21Bb}
                             	                     \end{align*}
                             	                     Therefore, using equations (\ref{eq:Pi1Bb}) and (\ref{eq:P21Bb}) we 
                             	                     have 
                             	                     \begin{align*}
                             	                       \sum_{l{=}1}^4 P_l(\boldsymbol{x}, t) 
                                           	           &{=} \underset{l {\ne} i, 2}{\sum_{l{=}1}^4} P_l(\boldsymbol{x}, t)
                             	                        {+} (0.01 P_i(\boldsymbol{x}, t{-} 1) {+} 0.99)\\
                             	                       &{+} (P_2(\boldsymbol{x}, t{-}1) {+} 0.99 P_i(\boldsymbol{x}, t{-}1)
                             	                        {-} 0.99)
                             	                       \\
                             	                       &{=} \underset{l {\ne} i, 2}{\sum_{l{=}1}^4} P_l(\boldsymbol{x}, t)
                             	                        {+} P_i(\boldsymbol{x}, t{-}1) {+} P_2(\boldsymbol{x}, t{-}1) \\
                             	                       &{=} \underset{l {\ne} i, 2}{\sum_{l{=}1}^4}P_l(\boldsymbol{x},t{-}1) 
                             	                        {+} P_i(\boldsymbol{x}, t{-}1) {+} P_2(\boldsymbol{x}, t{-}1) \\
                             	                       &{=} \sum_{l{=}1}^4 P_l(\boldsymbol{x}, t{-}1) {=} 1.
                             	                     \end{align*}
                             	\item[{\bf Case c:}] Assume $0 {<} P_i(\boldsymbol{x}, t{-}1) {+} \epsilon {<} 1$ 
                             	                     and $P_2(\boldsymbol{x}, t{-}1) {-} \epsilon {\le} 0$. Since\\
                             	                     $P_2(\boldsymbol{x}, t{-}1) {\le} \epsilon$ we have
                             	                     \begin{align*}
                             	                       &\text{sgn}(\epsilon) \min(|\Xi(\epsilon,P_i(\boldsymbol{x},t{-}1))|, 
                             	                                                  |\Xi({-}\epsilon,
                             	                                                       P_2(\boldsymbol{x},t{-}1))|)
                             	                       \\ 
                             	                       &{=} \min(|\epsilon|, |-0.99 P_2(\boldsymbol{x}, t{-}1)|) \\
                             	                       &{=} \min(\epsilon, 0.99 (1{-}P_2(\boldsymbol{x}, t{-}1))) 
                             	                        {=} 0.99 P_2(\boldsymbol{x}, t{-}1).
                             	                       \tag{14}
                             	                       \label{eq:incr1Bc}
                             	                     \end{align*}
                             	                     Using equation (\ref{eq:phenotypeMutate}) and considering 
                             	                     (\ref{eq:incr1Bc}) we have that
                             	                     \begin{align*}
                             	                       P_i(\boldsymbol{x}, t) &{=} P_i(\boldsymbol{x}, t{-}1)
                             	                        {+} 0.99 P_2(\boldsymbol{x}, t{-}1)
                             	                       \tag{15}
                             	                       \label{eq:Pi1Bc}
                             	                     \end{align*}
                             	                     Also using equation (\ref{eq:phenotypeMutateBalance}) and considering 
                             	                     again (\ref{eq:incr1Bc}) we have that
                             	                     \begin{align*}
                             	                       P_2(\boldsymbol{x}, t) &{=} P_2(\boldsymbol{x}, t{-}1) 
                             	                        {-} 0.99 P_2(\boldsymbol{x}, t{-}1) \\
                             	                       &{=} 0.01 P_2(\boldsymbol{x}, t{-}1).
                                       	               \tag{16}
                             	                       \label{eq:P21Bc}
                             	                     \end{align*}
                             	                     Therefore, using equations (\ref{eq:Pi1Bc}) and (\ref{eq:P21Bc}) we 
                             	                     have 
                             	                     \begin{align*}
                             	                       \sum_{l{=}1}^4 P_l(\boldsymbol{x}, t) 
                             	                       &{=} \underset{l {\ne} i, 2}{\sum_{l{=}1}^4} P_l(\boldsymbol{x}, t)
                             	                        {+} (P_i(\boldsymbol{x}, t{-} 1) {+} 0.99 P_2(\boldsymbol{x},t{-}1)) 
                             	                       \\
                             	                       &\hspace{0.25cm} {+} 0.01 P_2(\boldsymbol{x}, t{-}1) \\
                             	                       &{=} \underset{l {\ne} i, 2}{\sum_{l{=}1}^4} P_l(\boldsymbol{x}, t)
                             	                        {+} P_i(\boldsymbol{x}, t{-}1) {+} P_2(\boldsymbol{x}, t{-}1) \\
                             	                       &{=} \underset{l {\ne} i, 2}{\sum_{l{=}1}^4}P_l(\boldsymbol{x},t{-}1) 
                             	                        {+} P_i(\boldsymbol{x}, t{-}1) {+} P_2(\boldsymbol{x}, t{-}1) \\
                             	                       &{=} \sum_{l{=}1}^4 P_l(\boldsymbol{x}, t{-}1) {=} 1.
                             	                     \end{align*}
                                \item[{\bf Case d:}] Assume $P_i(\boldsymbol{x}, t{-}1) {+} \epsilon {\ge} 1$ 
                             	                     and $P_2(\boldsymbol{x}, t{-}1) {-} \epsilon {\le} 0$. 
                             	                     We compute 
                             	                     \begin{align*}
                             	                       &\text{sgn}(\epsilon) \min(|\Xi(\epsilon,P_i(\boldsymbol{x},t{-}1))|, 
                             	                                                  |\Xi({-}\epsilon, 
                             	                                                       P_2(\boldsymbol{x},t{-}1))|)
                             	                       \\ 
                             	                       &{=} \min(|0.99 (1{-}P_i(\boldsymbol{x}, t{-}1))|,
                             	                                 |{-}0.99 P_2(\boldsymbol{x}, t{-}1)|) \\
                             	                       &{=} \min(0.99 (1{-}P_i(\boldsymbol{x}, t{-}1)),
                             	                                 0.99 P_2(\boldsymbol{x}, t{-}1)).
                             	                       \tag{17}
                             	                       \label{eq:incr1Bd}
                             	                     \end{align*}
                             	                     There are two possible situations for the minimum in equation 
                             	                     (\ref{eq:incr1Bd}) but both are similar respectively to Case b and 
                             	                     c and so in either situation 
                                                     $\sum_{l{=}0}^4 P_l(\boldsymbol{x}, t) {=} 1$.
                             \end{itemize}
                           \end{itemize}
                           \item[{\bf Case 2:}] Now assume that $i {=} 2$ and $k {=} 1, 3, 4$ as we are balancing 
                                                modifications in the probability of quiescence equally against the
                                                probabilities of all the other phenotypic actions.
                           \begin{itemize}
                             \item[{\bf Case A:}] Assume $\epsilon {<} 0$.
                             \begin{itemize}
                               \item[{\bf Case a:}] Assume $0 {<} P_2(\boldsymbol{x}, t{-}1) {+} \epsilon {<} 1$.
                                                    We have that 
                                                    \begin{align*}
                                                      \Xi(\epsilon, P_2(\boldsymbol{x}, t{-}1)) {=} \epsilon
                                                      \tag{18}
                                                      \label{eq:incr2Aa}
                                                    \end{align*}
                                                    Using equation (\ref{eq:phenotypeMutate}) and considering 
                                                    (\ref{eq:incr2Aa}) we have that
                                                    \begin{align*}
                                                      P_2(\boldsymbol{x}, t) {=} P_2(\boldsymbol{x}, t{-}1) {+} \epsilon.
                                                      \tag{19}
                                                      \label{eq:P22Aa}
                                                    \end{align*}
                                                    Also using equation (\ref{eq:phenotypeMutateBalance}) and considering 
                                                    again (\ref{eq:incr2Aa}) we have that
                                                    \begin{align*}
                                                      P_k(\boldsymbol{x}, t) {=} P_k(\boldsymbol{x}, t{-}1) 
                                                                             {-} \frac{\epsilon P_k(\boldsymbol{x}, t{-}1)}
                                                                                      {\underset{m {\ne} 2}{\sum_{m{=}1}^4} 
                                                                                       P_m(\boldsymbol{x}, t{-}1)},
                                                      \tag{20}
                                                      \label{eq:Pk2Aa} \\
                                                      k {=} 1, 3, 4.
                                                    \end{align*}
                                                    Therefore, using equations (\ref{eq:P22Aa}) and (\ref{eq:Pk2Aa}) we have 
                                                    \begin{align*}
                                                      \sum_{l{=}1}^4 P_l(\boldsymbol{x}, t) 
                                                      &{=} \sum_{l{=}1}^4 P_l(\boldsymbol{x}, t{-}1)
                                                       {-} \epsilon \underset{l {\ne} 2}{\sum_{l{=}1}^4} 
                                                                      \frac{P_l(\boldsymbol{x}, t{-}1)}
                                                                           {\underset{m {\ne} 2}{\sum_{m{=}1}^4} 
                                                                           	P_m(\boldsymbol{x}, t{-}1)}
                                                       {+} \epsilon \\
                                                      &{=} \sum_{l{=}1}^4 P_l(\boldsymbol{x}, t{-}1)
                                                       {+} \epsilon \left (1 {-} \frac{\underset{l {\ne} 2}{\sum_{l{=}1}^4} 
                                                                                 P_l(\boldsymbol{x}, t{-}1)}
                                                                                {1 {-} P_2(\boldsymbol{x}, t{-}1)} \right) \\
                                                      &{=} \sum_{l{=}1}^4 P_l(\boldsymbol{x}, t{-}1)
                                                       {+} \epsilon \left (1 {-} \frac{1 {-} P_2(\boldsymbol{x}, t{-}1)}
                                                                                {1 {-} P_2(\boldsymbol{x}, t{-}1)} \right) \\
                                                      &{=} \sum_{l{=}1}^4 P_l(\boldsymbol{x}, t{-}1)
                                                       {+} \epsilon (1 {-} 1) \\
                                                      &{=} \sum_{l{=}1}^4 P_l(\boldsymbol{x}, t{-}1) {=} 1.
                                                    \end{align*}
                               \item[{\bf Case b:}] Assume $P_2(\boldsymbol{x}, t{-}1) {+} \epsilon {\le} 0$.
                                                    We have that 
                                                    \begin{align*}
                                                      \Xi(\epsilon, P_2(\boldsymbol{x}, t{-}1))
                                                      {=} {\minus}0.99 P_2(\boldsymbol{x}, t{-}1)
                                                      \tag{21}
                                                      \label{eq:incr2Ab}
                                                    \end{align*}
                                                    Using equation (\ref{eq:phenotypeMutate}) and considering 
                                                    (\ref{eq:incr2Ab}) we have that
                                                    \begin{align*}
                                                      P_2(\boldsymbol{x}, t) &{=} P_2(\boldsymbol{x}, t{-}1)
                                                                              {-} 0.99 P_2(\boldsymbol{x}, t{-}1)\\
                                                                             &{=} 0.01 P_2(\boldsymbol{x}, t{-}1).
                                                      \tag{22}
                                                      \label{eq:P22Ab}
                                                    \end{align*}
                                                    Also using equation (\ref{eq:phenotypeMutateBalance}) and considering 
                                                    again (\ref{eq:incr2Ab}) we have that
                                                    \begin{align*}
                                                      P_k(\boldsymbol{x}, t) {=} P_k(\boldsymbol{x}, t{-}1) 
                                                                             {+} \frac{0.99 P_2(\boldsymbol{x}, t{-}1)
                                                                             	       P_k(\boldsymbol{x}, t{-}1)}
                                                                                      {\underset{m {\ne} 2}{\sum_{m{=}1}^4} 
                               	                                                       P_m(\boldsymbol{x}, t{-}1)},
                                                      \tag{23}
                                                      \label{eq:Pk2Ab} \\
                                                      k {=} 1, 3, 4.
                                                    \end{align*}
                                                    Therefore, using equations (\ref{eq:P22Ab}) and (\ref{eq:Pk2Ab}) we have 
                                                    \begin{align*}
                                                      \sum_{l{=}1}^4 P_l(\boldsymbol{x}, t) 
                                                      &{=} \underset{l {\ne} 2}{\sum_{l{=}1}^4}  
                                                             P_l(\boldsymbol{x}, t{-}1) \\
                                                      &{+} 0.99 P_2(\boldsymbol{x}, t{-}1)
                                                           \underset{l {\ne} 2}{\sum_{l{=}1}^4}
                                                           \frac{P_l(\boldsymbol{x}, t{-}1)}
                                                                {\underset{m {\ne} 2}{\sum_{m{=}1}^4} 
                               	                                 P_m(\boldsymbol{x}, t{-}1)}\\
                                                      &{+} 0.01 P_2(\boldsymbol{x}, t{-}1) \\
                                                      &{=} \underset{l {\ne} 2}{\sum_{l{=}1}^4} P_l(\boldsymbol{x}, t{-}1)\\
                                                      &{+} 0.99 P_2(\boldsymbol{x}, t{-}1)
                                                           \frac{\underset{l {\ne} 2}{\sum_{l{=}1}^4}
                                                                 P_l(\boldsymbol{x}, t{-}1)}
                                                                {1 {-} P_2(\boldsymbol{x}, t{-}1)}
                                                       {+} 0.01 P_2(\boldsymbol{x}, t{-}1) \\
                                                      &{=} \underset{l {\ne} 2}{\sum_{l{=}1}^4} P_l(\boldsymbol{x}, t{-}1)\\
                                                      &{+} 0.99 P_2(\boldsymbol{x}, t{-}1)
                                                          \frac{1 {-} P_2(\boldsymbol{x}, t{-}1)}
                                                               {1 {-} P_2(\boldsymbol{x}, t{-}1)}
                                                       {+} 0.01 P_2(\boldsymbol{x}, t{-}1) \\
                                                      &{=} \underset{l {\ne} 2}{\sum_{l{=}1}^4} P_l(\boldsymbol{x}, t{-}1)
                                                       {+} 0.99 P_2(\boldsymbol{x}, t{-}1)
                                                       {+} 0.01 P_2(\boldsymbol{x}, t{-}1) \\
                                                      &{=} \underset{l {\ne} 2}{\sum_{l{=}1}^4} P_l(\boldsymbol{x}, t{-}1)
                                                       {+} P_2(\boldsymbol{x}, t{-}1)
                                                       {=} \sum_{l{=}1}^4 P_l(\boldsymbol{x}, t{-}1) {=} 1.
                                                    \end{align*}
                             \end{itemize}
                             \item[{\bf Case B:}] Assume $\epsilon {>} 0$.
                             \begin{itemize}
                             	\item[{\bf Case a:}] Assume $0 {<} 
                             	                             P_2(\boldsymbol{x}, t{-}1) {+} \epsilon 
                             	                               {<} 1$. 
                             	                     Same as Case (2, A, a) so 
                             	                     $$\sum_{l{=}1}^4 P_l(\boldsymbol{x}, t) {=} 1.$$
                             	\item[{\bf Case b:}] Assume $P_2(\boldsymbol{x}, t{-}1) {+} \epsilon {\ge} 1$.
                             	                     We have that 
                             	                     \begin{align*}
                             	                       \Xi(\epsilon, P_2(\boldsymbol{x}, t{-}1))
                             	                       {=} 0.99 (1 {-} P_2(\boldsymbol{x}, t{-}1))
                             	                       \tag{24}
                             	                       \label{eq:incr2Bb}
                             	                     \end{align*}
                             	                     Using equation (\ref{eq:phenotypeMutate}) and considering 
                             	                     (\ref{eq:incr2Bb}) we have that
                             	                     \begin{align*}
                             	                       P_2(\boldsymbol{x}, t) &{=} P_2(\boldsymbol{x}, t{-}1)
                             	                                               {+} 0.99 (1 {-} P_2(\boldsymbol{x}, t{-}1)\\
                             	                                              &{=} 0.01 P_2(\boldsymbol{x}, t{-}1) {+} 0.99.
                             	                       \tag{25}
                             	                       \label{eq:P22Bb}
                             	                     \end{align*}
                             	                     Also using equation (\ref{eq:phenotypeMutateBalance}) and considering 
                             	                     again (\ref{eq:incr2Bb}) we have that
                             	                     \begin{align*}
                             	                       P_k(\boldsymbol{x}, t) &{=} P_k(\boldsymbol{x}, t{-}1) 
                             	                                               {-} \frac{0.99 (1{-}P_2(\boldsymbol{x},t{-}1))
                             		                                                     P_k(\boldsymbol{x}, t{-}1)}
                             	                                                        {\underset{m{\ne}2}{\sum_{m{=}1}^4} 
                             		                                                     P_m(\boldsymbol{x}, t{-}1)}\\
                             	                      &{=} P_k(\boldsymbol{x}, t{-}1) \\ 
                             	                      &{+} \frac{(0.99 P_2(\boldsymbol{x}, t{-}1) 
                             	                                 {-} 0.99)
                             	                    	         P_k(\boldsymbol{x}, t{-}1)}
                             	                                {\underset{m{\ne}2}{\sum_{m{=}1}^4} 
                             	                                 P_m(\boldsymbol{x}, t{-}1)},
                             	                       \tag{26}
                             	                       \label{eq:Pk2Bb} \\
                             	                       k {=} 1, 3, 4.
                             	                     \end{align*}
                             	                     Therefore, using equations (\ref{eq:P22Bb}) and (\ref{eq:Pk2Bb}) we have 
                             	                     \begin{align*}
                             	                       \sum_{l{=}1}^4 P_l(\boldsymbol{x}, t) 
                             	                       &{=} \underset{l {\ne} 2}{\sum_{l{=}1}^4} P_l(\boldsymbol{x}, t{-}1)\\
                             	                       &{+} (0.99 P_2(\boldsymbol{x}, t{-}1) {-} 0.99)
                             	                            \underset{l {\ne} 2}{\sum_{l{=}1}^4}
                             	                                      \frac{P_l(\boldsymbol{x}, t{-}1)}
                             	                                           {\underset{m {\ne} 2}{\sum_{m{=}1}^4} 
                             		                                        P_m(\boldsymbol{x}, t{-}1)}\\
                             	                       &{+} 0.01 P_2(\boldsymbol{x}, t{-}1) {+} 0.99 \\
                                   	                   &{=} \underset{l {\ne} 2}{\sum_{l{=}1}^4} P_l(\boldsymbol{x}, t{-}1)\\
                             	                       &{+} (0.99 P_2(\boldsymbol{x}, t{-}1) {-} 0.99)
                                            	            \frac{\underset{l {\ne} 2}{\sum_{l{=}1}^4}
                             		                              P_l(\boldsymbol{x}, t{-}1)}
                             	                                 {1 {-} P_2(\boldsymbol{x}, t{-}1)} \\
                             	                       &{+} 0.01 P_2(\boldsymbol{x}, t{-}1) {+} 0.99 \\
                             	                       &{=} \underset{l {\ne} 2}{\sum_{l{=}1}^4} P_l(\boldsymbol{x}, t{-}1)\\
                             	                       &{+} (0.99 P_2(\boldsymbol{x}, t{-}1) {-} 0.99)
                             	                            \frac{1 {-} P_2(\boldsymbol{x}, t{-}1)}
                             	                                 {1 {-} P_2(\boldsymbol{x}, t{-}1)} \\
                                     	               &{+} 0.01 P_2(\boldsymbol{x}, t{-}1) {+} 0.99 \\
                             	                       &{=} \underset{l {\ne} 2}{\sum_{l{=}1}^4}  
                             	                              P_l(\boldsymbol{x}, t{-}1)
                             	                        {+} 0.99 P_2(\boldsymbol{x}, t{-}1) {-} 0.99\\
                             	                       &{+} 0.01 P_2(\boldsymbol{x}, t{-}1) {+} 0.99 \\
                             	                       &{=} \underset{l {\ne} 2}{\sum_{l{=}1}^4} P_l(\boldsymbol{x}, t{-}1)
                             	                        {+} P_2(\boldsymbol{x}, t{-}1)
                             	                        {=} \sum_{l{=}1}^4 P_l(\boldsymbol{x}, t{-}1) {=} 1.
                             	                     \end{align*}
                             \end{itemize}
                           \end{itemize}
	\end{itemize}
    Since the two cases above are exhaustive and all cases resulted in \newline
    $\sum_{l{=}1}^4 P_l(\boldsymbol{x}, t) {=} 1$ then 
    it must be that modifying phenotypic action $i$ with respect to some gene $j$ will 
    maintain the summation given by $\sum_{l{=}1}^4 P_l(\boldsymbol{x}, t) {=} 1$.
\end{proof}

\subsection{Update Rules for Cell Class}
The difference between the mutated class of a cell and the non-mutated class is 
that the mutated class has $\Upsilon {\in} \mathbb{N}$ positively mutated genes. 
Note that to delay the arrival of the first CSC not only does there have to be $\Upsilon$ positively mutated genes but for a SC or MNSC to transition into a CSC a random variable has to be less than the threshold $\iota {\in} \mathbb{R}_+(0, 1)$. This can be described by the following
\begin{align}
&\overline{\Psi}(\boldsymbol{x}, t, j) = 
\begin{cases}
   1, &\Psi(\boldsymbol{x}, t, j) {=} 1 \text{ or } \Psi(\boldsymbol{x}, t, j) {=} 2\\
   0, &\text{ otherwise}
\end{cases},
\label{eq:PositivelyMutatedIndicatorFunc}\\
&s(\boldsymbol{x}, t) {=} \begin{cases}
                 0 (\text{NTC}), & s(\boldsymbol{x}, t{-}1) {=} 1 (\text{MNTC}),\\
                  &\sum_{j=1}^G \overline{\Psi}(\boldsymbol{x}, t, j) {<} \Upsilon\\
                 \\
                 1 (\text{MNTC}), & s(\boldsymbol{x}, t{-}1) {=} 0 (\text{NTC}),\\
                  &\sum_{j=1}^G \overline{\Psi}(\boldsymbol{x}, t, j) {\ge} \Upsilon\\
                 \\
                 2 (\text{NSC}), & s(\boldsymbol{x}, t{-}1) {=} 3 (\text{MNSC}),\\
                  &\sum_{j=1}^G \overline{\Psi}(\boldsymbol{x}, t, j) {<} \Upsilon\\
                 \\
                 3 (\text{MNSC}), & s(\boldsymbol{x}, t{-}1) {=} 2 (\text{NSC}),\\
                  &\sum_{j=1}^G \overline{\Psi}(\boldsymbol{x}, t, j) {\ge} \Upsilon \\
                 \\
                 4 (\text{CSC}), & s(\boldsymbol{x}, t{-}1) {=} 2 (\text{NSC}),\\
                  &\sum_{j=1}^G \overline{\Psi}(\boldsymbol{x}, t, j) {\ge} \Upsilon,\\
                  & z {\le} \displaystyle \frac{\iota}{2}\\
                \\
                4 (\text{CSC}), &s(\boldsymbol{x}, t{-}1) {=} 3 (\text{MNSC}),\\
                  &\sum_{j=1}^G \overline{\Psi}(\boldsymbol{x}, t, j) {\ge} \Upsilon,\\
                  &z {\le} \iota
               \end{cases}
\label{eq:stateTransitionFunction},
\end{align}
where $z {\sim} U(0, 1)$, $\Psi(\boldsymbol{x}, t, j)$ is the mutation indicator function given by \eqref{eq:MutationFunc}, and $G$ is the number of genes.

Since each cell class has different probabilities for each phenotypic action, then 
there exists an initial phenotype matrix, $\tilde{P} {\in} \mathbb{R}^{6{\times}4}(0, 1)$ 
with $\sum_{i{=}1}^{4} \tilde{P}_{ki} {=} 1$, $k {=} 0, 1, 2, 3, 4, 5$.
Therefore, whenever a non-empty cell changes class, the phenotype vector 
must be set to its' new cell class's initial values, while also keeping all the 
changes that have happened to the phenotype vector. This is achieved through the following process
\begin{align}
&P_i(\boldsymbol{x}, t) {=} \tilde{P}_{ki} {+} \Xi(P_i(\boldsymbol{x}, t{-}1)
                                                      {-} \tilde{P}_{s(\boldsymbol{x}, t{-}1)i},
                                                          \tilde{P}_{ki}), 
\label{eq:PhenotypeTransferFunc}\\
& \text{where } k \text{ is the new cell state}, i {=} 1, 2, 3, 4.\nonumber 
\end{align}
If the probabilities of the phenotypic actions no longer sum to one after the changes, then standard normalization is used to fix the sum.

\subsection{Dedifferentiation}
Dedifferentiation is the process of a specialized cell reverting back to a non-specialized cell. In our model this is accomplished by a non stem cell becoming a stem cell. Dedifferentiation is used to help maintain the proper ratio of stem cells to non stem cells in the grid by dedifferentiating whenever the number of stem cells in the neighbourhood of a non stem cell is less than or equal to some chosen value, $\hat{S}$, or if the number of empty cells in the neighbourhood of a non stem cell is less than or equal to some chosen value, $\hat{E}$, or by random chance based on a sample from the uniform distribution being less than a threshold. To help reduce the number of cells dedifferentiating, the process is completed only when a random sample from the uniform distribution is less than or equal to some threshold, $\hat{D} {\in} \mathbb{R}_+(0, 1)$. If the non stem cell is not a TAC then the threshold is set at $\frac{\hat{D}}{2}$. In the case of random chance the threshold is set at $\frac{\hat{D}}{4}$. This process is represented by the function:
\begin{align}
s(\boldsymbol{x}, t) {=} \begin{cases}
                 2 (\text{NSC}), & s(\boldsymbol{x}, t{-}1) {=} 0 (\text{NTC})
                 \\
                 3 (\text{MNSC}), & s(\boldsymbol{x}, t{-}1) {=} 1 (\text{MNTC})
                 \\
                 4 (\text{CSC}), & s(\boldsymbol{x}, t{-}1) {=} 5 (\text{TC})
               \end{cases}.
\label{eq:DeDifferentiationFunction}
\end{align}

\subsection{Cell Fitness}
Each cell has a fitness value associated to it, so that the cells can compete and
the population contains only the healthiest, or in the case of mutated cells,
the most positively mutated cells. The characteristics that affect the fitness are
based upon work by \textcite{Bowling}, in which they point to the following
important characteristics:
\begin{enumerate}
\item if a cell has a high apoptotic rate, it is less fit;
\item if a cell has a high proliferation rate, it is more fit;
\item if a cell is older, then it is less fit;
\item if a gene in a cell is mutated towards cancer, than it is less fit unless
      the cell is cancerous, in which case it is more fit;
\item if a cell is what they call super-competitive, then it is more fit than any type of cell.
\end{enumerate}
Based upon this the fitness of a cell in the CA is computed using the functions $\hat{M}$ which is the gene expression ratio and the fitness function $\hat{F}$:
\begin{align}
\hat{M}(\boldsymbol{x}, t, j) &{=} \begin{cases}
                                     {-}\displaystyle \frac{E_j(\boldsymbol{x}, t)}{\overline{M}} 
                                         &, (T_j = 0 \text{ and } E_j(\boldsymbol{x}, t) < 0)\\
                                         &\text{ or } (T_j = 1 \text{ and } E_j(\boldsymbol{x}, t) > 0) \\ \\
                                    \displaystyle \frac{E_j(\boldsymbol{x}, t)}{\overline{M}}
                                         &, (T_j = 0 \text{ and } E_j(\boldsymbol{x}, t) > 0)\\
                                         &\text{ or } (T_j = 1 \text{ and } E_j(\boldsymbol{x}, t) < 0) \\ \\
                                    0 &, \text{otherwise}
                                \end{cases},\label{eq:GeneExpressionRatio}\\
\hat{F}(\boldsymbol{x}, t) &{=} \begin{cases}
                            \displaystyle \frac{P_1(\boldsymbol{x}, t)}
                                               {\tilde{P}_{s(\boldsymbol{x}, t)1}} 
                              &, \overline{\tau}(\boldsymbol{x}, t) {=} 0\\
                            \\
                            \displaystyle \frac{P_1(\boldsymbol{x}, t)}
                                               {\tilde{P}_{s(\boldsymbol{x}, t)1} {+} \omega} 
                              &, \overline{\tau}(\boldsymbol{x}, t) {=} 1
                          \end{cases} \nonumber \\
                          &{-} \frac{P_3(\boldsymbol{x}, t)}{\tilde{P}_{s(\boldsymbol{x}, t)3}}
                           {-} \alpha(\boldsymbol{x}, t)P_3(\boldsymbol{x}, t)
                          \label{eq:FitnessFunc} \\
                          &{+} \begin{cases}
                                 \sum_{j{=}1}^{G} \hat{M}(\boldsymbol{x}, t, j) 
                                   &, s(\boldsymbol{x}, t) {=} 0 \text{ (NTC) or } 
                                      s(\boldsymbol{x}, t) {=} 2 (\text{NSC})\\
                                 \\
                                 {-}\sum_{j{=}1}^{G} \hat{M}(\boldsymbol{x}, t, j) 
                                   &, \text{otherwise}\\
                          \end{cases}\nonumber,
\end{align}
where $E_j(\boldsymbol{x}, t)$ is the gene-expression of gene $j$, $\overline{M}$ is the threshold that determines where gene $j$ is mutated, $T_j$ indicates if the gene $j$ is either a tumour suppressor gene or oncogene, $P_1(\boldsymbol{x}, t)$ is the probability that proliferation will occur in a time-step, $P_3(\boldsymbol{x}, t)$ is the probability that apoptosis will occur in a time-step, $\alpha(\boldsymbol{x}, t)$ is the age of the cell, $\overline{\tau}(\boldsymbol{x}, t)$ indicates if the cell is a TAC or not, $\omega$ is the amount the probability of proliferation is increased when a cell is a TAC, and $\Tilde{P}$ is the initial phenotype matrix that provides the initial values for each of the phenotypic actions for each cell class.
If the cell is from a SC class (SC, MNSC, CSC) then its' fitness is multiplied by a factor $\Pi {\in} \mathbb{R}_+$ if $\hat{F}(\boldsymbol{x}, t) {>} 0$ or $\Pi^{-1}$ if $\hat{F}(\boldsymbol{x}, t) {<} 0$, so that a SC type cell has a higher fitness than a non-stem cell.

\subsection{CA Rule}
Each non-empty cell in the CA grid chooses a phenotypic action to execute for that 
time-step and attempts to complete such action. The phenotypic action is chosen by 
taking a random sample from the probability distribution that is generated by the 
phenotype vector. Consider that the cell that is performing the phenotypic action
is located at $\boldsymbol{x}^{(p)} {\in} \Omega$. The cell the action is being applied to 
will be located at $\boldsymbol{x}^{(c)} {\in} \Omega$, which is a randomly chosen 
location in the cell's neighbourhood. The randomly chosen cell is an empty cell or a cell with a lower fitness in the case of proliferation, and differentiation. It is important to note that only NSC, MNSC, and CSC can differentiate.

CSCs and TCs are the only class of cells that can kill other cells when moving during quiescence. If the parent cell is a CSC or TC and the chosen cell has a higher fitness then the phenotypic action is accomplished only if a sample from some random variable 
is less than a threshold to kill, $\kappa {\in} \mathbb{R}_+(0, 1)$. A CSC can kill a TC and TC a CSC only if the fitness is lower, as otherwise the tumour cell population and/or CSC population would die out before they have had the chance to thrive. 

\subsubsection{Proliferation}
Assuming all the conditions for the phenotypic action to occur are met then 
the following changes occur to the states of the parent and child cell upon proliferation:
\begin{align}
&\boldsymbol{S}(\boldsymbol{x}^{(k)}, t) {=} \begin{cases}
                     [ s(\boldsymbol{x}^{(p)}, t{-}1), 0, 
                       \boldsymbol{E}(\boldsymbol{x}^{(p)}, t{-}1),
                       \boldsymbol{P}(\boldsymbol{x}^{(p)}, t{-}1),
                       & ,\overline{\tau}(\boldsymbol{x}^{(p)}, t{-}1) {=} 0 \\
                     \hspace{0.25cm} \overline{\tau}(\boldsymbol{x}^{(p)}, t{-}1), 
                     \overline{n}(\boldsymbol{x}^{(p)}, t{-}1) ] \\
                     \\
                     [ s(\boldsymbol{x}^{(p)}, t{-}1), 0,
                        \boldsymbol{E}(\boldsymbol{x}^{(p)}, t{-}1),
                        \boldsymbol{P}(\boldsymbol{x}^{(p)}, t{-}1),
                       & ,\overline{\tau}(\boldsymbol{x}^{(p)}, t{-}1) {=} 1, \\
                       \hspace{0.25cm} \overline{\tau}(\boldsymbol{x}^{(p)}, t{-}1),
                       \overline{n}(\boldsymbol{x}^{(p)}, t{-}1){+}1 ]
                       & \overline{n}(\boldsymbol{x}^{(p)}, t{-}1) {\ne} \Theta \\
                     \\
                     [ s(\boldsymbol{x}^{(p)}, t{-}1), 0,
                        \boldsymbol{E}(\boldsymbol{x}^{(p)}, t{-}1),
                       & , \overline{\tau}(\boldsymbol{x}^{(p)}, t{-}1) {=} 1, \\
                     \hspace{0.25cm} [ P_1(\boldsymbol{x}^{(p)}, t{-}1) {-} \Xi(\omega, P_1(\boldsymbol{x}^{(p)}, t{-}1)),
                     & \overline{n}(\boldsymbol{x}^{(p)}, t{-}1) {=} \Theta \\
                     \hspace{0.4cm}  P_2(\boldsymbol{x}^{(p)}, t{-}1) {+} \Xi(\omega, P_2(\boldsymbol{x}^{(p)}, t{-}1)),\\
                       \hspace{0.4cm} P_3(\boldsymbol{x}^{(p)}, t{-}1), 
                       P_4(\boldsymbol{x}^{(p)}, t{-}1) ], 0, 0 ]
                    \end{cases}, \label{eq:ProliferationStateChange}
\end{align}
where $k {\in} \{ p, c \}$; $\omega {\in} \mathbb{R}_+(0, 1)$; $\boldsymbol{S}(\boldsymbol{x}, t)$ is defined by equation \eqref{eq:StateVector}; $\Xi(\epsilon, \xi)$, defined by equation \eqref{eq:phenotypicActionBoundaryFunction}, ensures that the increment $\omega$ does not cause the phenotypic action being modified to leave the interval $[0, 1]$.

\subsubsection{Quiescence}
Assuming all the conditions for the phenotypic action to occur are met then the 
following changes occur to the states of the parent and child cell upon cell movement:
\begin{align}
&\boldsymbol{S}(\boldsymbol{x}^{(p)}, t) {=} \begin{cases}
                     [ 6, 0, \boldsymbol{0}, \boldsymbol{0}, 0, 0 ]
                       &, z {<} \zeta \\
                       \\
                       [ s(\boldsymbol{x}^{(p)}, t{-}1), \alpha(\boldsymbol{x}^{(p)}, t{-}1), 
                        \boldsymbol{E}(\boldsymbol{x}^{(p)}, t{-}1),
                       &, \text{otherwise} \\
                       \hspace{0.25cm} \boldsymbol{P}(\boldsymbol{x}^{(p)}, t{-}1),
                       \overline{\tau}(\boldsymbol{x}^{(p)}, t{-}1),
                       \overline{n}(\boldsymbol{x}^{(p)}, t{-}1) ]
                    \end{cases}, \label{eq:QuiesenceStateChangeParent}\\
&\boldsymbol{S}(\boldsymbol{x}^{(c)}, t) {=} \begin{cases}
                     [ s(\boldsymbol{x}^{(p)}, t{-}1), \alpha(\boldsymbol{x}^{(p)}, t{-}1), 
                        \boldsymbol{E}(\boldsymbol{x}^{(p)}, t{-}1),
                       &, z {<} \zeta \\
                       \hspace{0.25cm} \boldsymbol{P}(\boldsymbol{x}^{(p)}, t{-}1),
                       \overline{\tau}(x_1, y_1, t{-}1, \overline{n}(\boldsymbol{x}^{(p)}, t{-}1) ]\\
                       \\
                       [ s(\boldsymbol{x}^{(c)}, t{-}1), \alpha(\boldsymbol{x}^{(c)}, t{-}1), 
                        \boldsymbol{E}(\boldsymbol{x}^{(c)}, t{-}1),
                       &, \text{otherwise} \\
                      \hspace{0.25cm} \boldsymbol{P}(\boldsymbol{x}^{(c)}, t{-}1),
                      \overline{\tau}(\boldsymbol{x}^{(c)}, t{-}1), \overline{n}(\boldsymbol{x}^{(c)}, t{-}1) ]
                    \end{cases}, \label{eq:QuiesenceStatechangeChild}
\end{align}
where $z {\sim} U(0, 1)$ and $\zeta {\in} \mathbb{R}_+(0, 1)$ is the threshold that must be met 
for the cell to move.

\subsubsection{Apoptosis}
The following changes occur to the state of the cell at $\boldsymbol{x}^{(p)}$ upon apoptosis:
\begin{align}
&\boldsymbol{S}(\boldsymbol{x}^{(p)}, t) {=} [ 6, 0, \boldsymbol{0}, \boldsymbol{0}, 0, 0 ].
\label{eq:ApoptosisStateChange}
\end{align}

\subsubsection{Differentiation}
Assuming all the conditions for the phenotypic action to occur are met then the  
following changes occur to the states of the parent and child cell upon differentiation:
\begin{align}
\boldsymbol{S}(\boldsymbol{x}^{(p)}, t) &{=} [ s(\boldsymbol{x}^{(p)}, t{-}1), 0, 
                                               \boldsymbol{E}(\boldsymbol{x}^{(p)}, t{-}1),
                                               \label{eq:DifferentiationStateChangeParent}\\
                                               &\hspace{0.4cm} \boldsymbol{P}(\boldsymbol{x}^{(p)}, t{-}1), \overline{\tau}(\boldsymbol{x}^{(p)}, t{-}1), 
                                               \overline{n}(\boldsymbol{x}^{(p)}, t{-}1) ]
\nonumber\\
\overline{\Sigma}(\boldsymbol{x}^{(p)}, t) &{=} \begin{cases}
	0 (\text{{\footnotesize NTC}}) &, s(\boldsymbol{x}^{(p)}, t{-}1) {=} 2 (\text{{\footnotesize NSC}})\\
    1 (\text{{\footnotesize MNTC}}) &, s(\boldsymbol{x}^{(p)}, t{-}1) {=} 3 (\text{{\footnotesize MNSC}})\\
    5 (\text{{\footnotesize TC}}) &, s(\boldsymbol{x}^{(p)}, t{-}1) {=} 4 (\text{{\footnotesize CSC}})
  \end{cases} \label{eq:SCChildIndicatorFunction}\\
\boldsymbol{S}(\boldsymbol{x}^{(c)}, t) &{=} 
                     [ \overline{\Sigma}(\boldsymbol{x}^p, t), 0, 
                       \boldsymbol{E}(\boldsymbol{x}^{(p)}, t{-}1), \label{eq:DifferentiationStateChangeChild}\\
                       &\hspace{0.4cm} [ P_1(\boldsymbol{x}^{(p)}, t{-}1) {+} \Xi(\omega, P_1(\boldsymbol{x}^{(p)}, t{-}1)),\nonumber\\
                       &\hspace{0.4cm} P_2(\boldsymbol{x}^{(p)}, t{-}1) {-} \Xi(\omega, P_2(\boldsymbol{x}^{(p)}, t{-}1)),\nonumber \\
                       &\hspace{0.4cm} P_3(\boldsymbol{x}^{(p)}, t{-}1), P_4(\boldsymbol{x}^{(p)}, t{-}1) ], 1, 0 ]
\nonumber,
\end{align}
where $\omega {\in} \mathbb{R}_+(0, 1)$ is the amount the probability of proliferation changes when 
the cell is a TAC.

\subsection{Tumour Excision}
For the purpose of exploring the length of time it takes for a tumour to recur, we include three simple types of excisions in the model. One whereby, all the tumour cells in the grid and their neighbours up-to a certain depth are removed. Another, where all the mutated cell class cells and their neighbours up-to a certain depth are removed. Lastly, where a circular sub-region of the grid is removed. 

\subsection{Lineage Tracking}
The lineage of each cell is tracked for the purpose of following tumour cell lineages from their origin, checking how many independent tumour masses form throughout the simulation, and whether the origin is monoclonal or polyclonal. The lineage is tracked using the following methods. Each cell has a parameter
$\Lambda(\boldsymbol{x}, t) {\in} \{ {-}1, 0, 1, ..., L {\times} M {-} 1 \}$, where $L$ is the length in the x-direction and $M$ is the length in the y-direction, that when set to ${-}1$ implies no information is known or the cell is empty, and if set to any other value represents the spatial index of the parent at the time it created the first cell in the lineage. 

During cell movement the parameter is passed down from parent ($\boldsymbol{x}^{(p)}$) to child cell ($\boldsymbol{x}^{(c)}$). In the case of apoptosis the parameter is set to ${-}1$. Finally, during proliferation and differentiation the parameter is set as follows for the cells $\boldsymbol{x}^{(p)}$ and $\boldsymbol{x}^{(c)}$:
\begin{align}
&\Lambda(\boldsymbol{x}, t) {=} \begin{cases}
                           x_1^{(p)} M {+} x_2^{(p)},
                            & \Lambda(\boldsymbol{x}^{(p)}, t{-}1) {=} -1 \\
                           \Lambda(\boldsymbol{x}^{(p)}, t{-}1), 
                            & \text{otherwise}
                          \end{cases}.\label{eq:LineageTracker}\\
\end{align}

\subsection{Field Definition}
A field in this model is defined to be any groupings of mutated cells, thus a field is a section of the grid that contains few or no NTC or NSC. 
We can also further define a field based around the lineages, so that a grouping of mutated cells that share the same lineage would be considered a field. 

\subsection{CA Model Timeline}
Each time-step has the following order of actions:
\begin{enumerate}
    \item update the carcinogen sensitivity function,
    \item run the gene expression neural network,
    \item update the gene expressions based upon the output of the neural network in step 2,
    \item update the gene expressions via the gene instability process,
    \item update the phenotype vector based upon the gene expressions of each gene,
    \item update the states of each cell using the state transition process,
    \item apply the dedifferentiation process,
    \item apply the phenotypic action chosen by the cell for that time-step,
    \item possibly perform tumour excision.
\end{enumerate}

\end{document}